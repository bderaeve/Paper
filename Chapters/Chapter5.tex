% Chapter 5
%----------------------------------------------------------------------------------------
\section{Extracting the data}
\label{extracting}
\subsection{Programming language choice}
To start programming a choice had to be made. Which programming language and what operating system? Since we only knew Java, C++ and C these languages had our preference. Another requirement was that running the program on an embedded device. Linux has more options is the embedded domain then windows. For instance the raspberry pi is a cheap embedded device with an Ethernet connection and 2 usb ports and a 700mhz ARM processor that can run different linux distributions.\\
The easiest language is definitely java. It is easy because a lot of libraries are already included in the language. For a gateway several libraries are necessary. A HTTP library to send requests to ipsum and receive requests directly from the web application. Second, an XML or JSON library to format data in these HTTP requests. Since the cloud storage system, Ipsum, uses XML the rest of the gateway also uses XML instead of JSON.\\
On the other side there is the zigbee network which communicates using RS232. In java this serial communication is also build in. In C or C++ it depends on the OS, for linux it is possible to use read and write to and from file descriptors.\\
So on first sight java seems to be the preferred choice. But since it has to run embedded and processing power might be limited java is less ideal. On a raspberry pi computing power is no problem. Maybe in the future more services should run on the same device so there is an advantages in using C or C++.\\
So this leaves C and C++. They are both equally fast in execution but C++ is easier to develop since  it support more libraries (libraries for C van also be used in C++ but not the other way around) and it has a lot more typechecking so that code that compiles will probably also execute without errors. In C there is a lot of static casting to void pointers and this often results in harder to find bugs. C++ also supports objects and classes that make it easier to create a larger program. There are several other advantages in C++ over C. An entire chapter is dedicated to the advantages of C++ over C in the book “thinking in C++” by Bruce Eckel. Also we wanted to have some more experience in C++ and writing this program in C++ sure improved our programming.
%-------------------------------------------------------------------
\subsection{Implementation aspects}
For the gateway it took some time to come up with the correct structure. Since a gateway generally has to wait a lot for incoming messages on different connections it is logical to construct a lot of threads which can wait while other threads keep on running. An option would be to use the pipeline pattern where 1 thread “generates” packets. This would be the zigbee receiver or the webservice. Then a few threads would act as filters on these packets, doing the necessary processing.\\
One simple pipe would not work since there are more threads generating packets. For instance the ipsum thread can also generate packets to indicate that some packets could not be processed correctly. If ipsum is down this thread needs to store messages in the sql database.\\
There is not really one flow of information going from one	 place to another. Information is coming in at different places and is also leaving the program at different places.\\
A more flexible pattern is the thread pool pattern. This pattern is often used for webservers. Each thread then handles an incoming connection. Although mongoose uses this pattern in this way, it is not entirely the same for the gateway program. The gateway has a few threads, one for each type of connection. 
%---------------------------------
\subsubsection{Ipsum}
There is one thread for ipsum where the connection to the ipsum database is maintained. Uploading data, changing frequency or in use setting of sensors are all tasks this thread has. The ipsum thread has an incoming queue and an outgoing queue. The incoming queue can contain several packets to indicate what data should be uploaded, what sensor frequencies should change or what nodes should be activated. If for some reason something goes wrong, ipsum will push packets on the outgoing queue to the main thread.
%---------------------------------
\subsubsection{ZigBee}
On the zigbee side there are 2 threads, one for sending and one for receiving packets. Received packets are pushed onto a queue that is read out in the main thread. Received packets can be sensor data, errors and responses from sent packets. When the main thread needs something to be done in the zigbee network it will push a packet onto the queue going to the zigbee sender thread. For instance: changes sample frequencies, request IO data, activate a node.
%---------------------------------
\subsubsection{Web service}
As mentioned before, mongoose is used as webserver. Mongoose sets up some threads to handle incoming connections. All these threads can push packets onto a queue going to the main thread. In the main threads the XML data is analyzed and the right actions are performed. Possible webservice requests are: add node, add sensor, request IO and change frequency.
%---------------------------------
\subsubsection{Main thread}
All intelligence can be found in the main thread. There it is decided what to do with incoming packets. Most often this means creating packets and putting them in the appropriate queues. \\
The main thread also has access to local storage in the form of an sqlite database. Node information has to be stored locally in order to link zigbee addresses to the correct Ipsum ID’s. When a sensor data packet is received the main needs to look up to what ipsum sensors it has to upload this data. The relation between Ipsum ID and zigbee address is made when the webservice receives add node and add sensor requests.\\
The local database also stores Ipsum packets that could not be sent. This could happen when the connection to Ipsum is down or the Ipsum service has crashed. Since losing data is not acceptable, it is also stored in in the sql database.\\
Also errors could be logged in the database. For now errors are printed to \verb+cerr+. 
%------------------------------------------------------------------
\subsection{Hardware}
The gateway consists of a zigbee radio connected to a computer via RS232. This computer could evolve into an embedded linux device such as the raspberry pi (http://www.raspberrypi.org/). It is important that it runs on linux since some libraries are necessary and the serial communication is based on system calls to the kernel. For instance xerces for XML parsing or boost for the multithreading and some other small features.\\
The library for the webserver is mongoose and the one for en sql database is sqlite. Both libraries are written in C and consist out of 1 header and 1 source file and are both compiled into the final program. So unlike xerces and boost the OS will not need to install these libraries since they are not dynamically linked.\\
RS232 is a simple serial protocol that is used for low data rates. In linux an RS232 connection is easily set up by opening a file descriptor with the necessary options to configure baud rate, parity, stop bits, etc. After that you can use read and write functions to receive and send data from the zigbee radio.

