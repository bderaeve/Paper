% Appendix B

\chapter{Matlab code Equaliser} % Main appendix title

\label{AppendixB} % For referencing this appendix elsewhere, use \ref{AppendixA}

\lhead{Appendix B. \emph{Matlab code Equaliser}} % This is for the header on each page - perhaps a shortened title

\begin{lstlisting}[language=Matlab]
% 3 BAND EQ TEST WITH WHITE GAUSSIAN NOISE AND FAST FOUTIER TRANSFORM
 
% Filter settings
 
Fs = 48000;              % Sampling Frequency
N    = 100;              % Order
win = hamming(N+1);      % Hamming window
flag = 'scale';          % Sampling Flag
 
LpGain = 1;              % set between 0 and 1 for the lowpass gain (0-800Hz)
BpGain = 1;              % set between 0 and 1 for the bandpass gain (800 - 6000Hz)
HpGain = 0;              % set between 0 and 1 for the highpass gain (6000 - 24000Hz)
 
% LP coefficients
 
Fc   = 800;              % Cutoff Frequency
blp  = fir1(N, Fc/(Fs/2), 'low', win, flag);
 
% HP coefficients
 
Fc   = 6000;             % Cutoff Frequency
bhp  = fir1(N, Fc/(Fs/2), 'high', win, flag);
 
% BP coefficients
 
Fc1  = 800;   %800           % First Cutoff Frequency
Fc2  = 6000;  % 6000          % Second Cutoff Frequency
bbp  = fir1(N, [Fc1 Fc2]/(Fs/2), 'bandpass', win, flag);
 
 
 
% Creation of the main filtercoefficients
filter = zeros(1,N+1);
 
for i=1:N+1
value = bhp(i)*HpGain + blp(i)*LpGain + bbp(i)*BpGain;
filter(i) = value;
end
 
% Creation of a whitenoise test block
 
data = wgn(1,1024,1); % generates array of 1024 white noise samples
 
% Proces the data with the filter
 
outputarray = zeros(1, length(filter)+length(data)-1);
 
 
% adjusts the inputed data, puts a series of zeros at the front and the end
adjdata = [zeros(1,length(filter)-1) data zeros(1,length(filter))];
 
for n=length(filter):length(data)+length(filter)+1,
    outputvalue = 0;
   
 	for k=1:length(filter),   
 
 	outputvalue = outputvalue + adjdata(n-k+1)* filter(k);
 	end;
 	
    outputarray(n-length(filter)+1) = outputvalue;
    
end;
 
 
% PLOT FFT
L = 1024
NFFT = 2^nextpow2(L); % Next power of 2 from length of y
Y = fft(outputarray,NFFT)/L;
 
f = Fs/2*linspace(0,1,NFFT/2+1);
 
 
% Plot single-sided amplitude spectrum.
q = 2*abs(Y(1:NFFT/2+1));
plot(f,q) 
title('Single-Sided Amplitude Spectrum of the outputsignal')
xlabel('Frequency (Hz)')
ylabel('|Y(f)|')

\end{lstlisting}
