%----------------------------------------------------------------------------------------
%	PACKAGES AND OTHER DOCUMENT CONFIGURATIONS
%----------------------------------------------------------------------------------------

\documentclass[10pt, a4paper, oneside]{Thesis} % Paper size, default font size and one-sided paper
\graphicspath{{./Pictures/}} % Specifies the directory where pictures are stored
%\usepackage[square, numbers, comma, sort&compress]{natbib} % Use the natbib reference package - read up on this to edit the reference style; if you want text (e.g. Smith et al., 2012) for the in-text references (instead of numbers), remove 'numbers' 
\usepackage{stmaryrd}
\usepackage{epsfig}
\usepackage{alltt}
\usepackage{listings}
\usepackage{xr-hyper}
\usepackage{hyperref}
\usepackage{float}
\usepackage{array} 
%\usepackage[nottoc,numbib]{tocbibind}

\hypersetup{urlcolor=black, colorlinks=true} % Colors hyperlinks in blue - change to black if annoying
\title{\ttitle} % Defines the thesis title - don't touch this

\begin{document}
\frontmatter % Use roman page numbering style (i, ii, iii, iv...) for the pre-content pages
\setstretch{1.3} % Line spacing of 1.3

% Define the page headers using the FancyHdr package and set up for one-sided printing
\fancyhead{} % Clears all page headers and footers
\rhead{\thepage} % Sets the right side header to show the page number
\lhead{} % Clears the left side page header

\pagestyle{fancy} % Finally, use the "fancy" page style to implement the FancyHdr headers

\newcommand{\HRule}{\rule{\linewidth}{0.5mm}} % New command to make the lines in the title page

%----------------------------------------------------------------------------------------
%	TITLE PAGE
%----------------------------------------------------------------------------------------
\begin{titlepage}
\begin{center}
\includegraphics{Logo}\\[1.5cm] % University/department logo
\textsc{\LARGE \univname}\\[2cm] % University name
%\textsc{\LARGE "Thesis"}\\[1.5cm]

\HRule \\[0.5cm] % Horizontal line
\begin{spacing}{2.0}
{\huge \bfseries \ttitle}\\[0.1cm] % Thesis title
\end{spacing}
\HRule \\[1.0cm] % Horizontal line

\includegraphics[height=6cm]{waspmote} \\[1.0cm] 

\begin{minipage}{0.4\textwidth}
\begin{flushleft} \large
\emph{Authors:}\\
{\authornames}
\end{flushleft}
\end{minipage}
\begin{minipage}{0.4\textwidth}
\begin{flushright} \large
\emph{Supervisor:} \\
{\supname}
\end{flushright}
\end{minipage}\\[2cm]

{\large \today}\\% Date

\vfill
\end{center}
\end{titlepage}

%----------------------------------------------------------------------------------------
%	REAL TITLE PAGE
%	Your institution may give you a different text to place here
%----------------------------------------------------------------------------------------

\addtocontents{toc}{\vspace{1em}} % Add a gap in the Contents, for aesthetics
\thispagestyle{empty}
\null\vfill 

\begin{center}
\Large{\textbf{DESIGN OF A WIRELESS SENSOR NETWORKING TEST-BED}}\\
\vspace{3cm}
\large{\emph{Bjorn Deraeve $\quad\bullet\quad$ Roel Storms}} \\
\HRule \\[1.0cm]
\vspace{1cm}
\normalsize{Written on behalf of Mr. Luc Vandeurzen}\\
\vspace{12cm}
{\large \today}
\end{center}

\vfill\vfill\vfill\vfill\

\clearpage % Start a new page

%----------------------------------------------------------------------------------------
%	QUOTATION PAGE
%----------------------------------------------------------------------------------------

%\pagestyle{empty} % No headers or footers for the following pages
%\null\vfill % Add some space to move the quote down the page a bit
%\textit{``Be a yardstick of quality. Some people aren't used to an environment where excellence is expected."}
%\begin{flushright}
%Steve Jobs
%\end{flushright}
%\vfill\vfill\vfill\vfill\vfill\vfill\null % Add some space at the bottom to position the quote just right
%\clearpage % Start a new page

%----------------------------------------------------------------------------------------
%	ABSTRACT PAGE
%----------------------------------------------------------------------------------------

\addtotoc{Abstract} % Add the "Abstract" page entry to the Contents

\abstract{\addtocontents{toc}{\vspace{1em}} % Add a gap in the Contents, for aesthetics

\emph{Audio applications are on of the main subfields of digital signal processing. More specific, in audio and music one can use frequency modulation synthesis in order to make creative and totally different sounding tones. A theoretical background and practical approach with an Analog Devices DSP-board is reported here.\\ \\
In the first two chapters it is explained how sound works and what aspects of frequency modulation are actually responsible for making a particular tone sound differently. Next a practical approach to add envelopes and effects to the tone is given.\\ \\
Chapter \ref{Chapter5} serves as introduction to \ref{Chapter6}. In this chapter some theoretical background and electronic designs of CPU architectures are explained. This chapter serves as a backbone for the practical implementations explored in chapter \ref{Chapter6}. Readers with a basic knowledge on this matter can skip this chapter and immediately start with chapter 6. The first part of this chapter handles the Analog Devices ADSP-21369 DSP board's initialisation routines and the second part focusses on the practical implementation of the signal creation and modulation algorithms.\\ \\
Finally chapter \ref{Chapter7} deals with the communication between the DSP-board and the Labview interface for the FM Synthesizer.\\ \\
As a conclusion chapter \ref{Chapter8} closes this report with some critical reflections on the implementation and team activities.}
}

\clearpage % Start a new page

%----------------------------------------------------------------------------------------
%	ACKNOWLEDGEMENTS
%----------------------------------------------------------------------------------------

%\setstretch{1.3} % Reset the line-spacing to 1.3 for body text (if it has changed)
%\acknowledgements{\addtocontents{toc}{\vspace{1em}} % Add a gap in the Contents, for aesthetics
%The acknowledgements and the people to thank go here, don't forget to include your project advisor\ldots
%}
%\clearpage % Start a new page

%----------------------------------------------------------------------------------------
%	LIST OF CONTENTS/FIGURES/TABLES PAGES
%----------------------------------------------------------------------------------------

\pagestyle{fancy} % The page style headers have been "empty" all this time, now use the "fancy" headers as defined before to bring them back

\lhead{\emph{Contents}} % Set the left side page header to "Contents"
\tableofcontents % Write out the Table of Contents

\lhead{\emph{List of Figures}} % Set the left side page header to "List of Figures"
\listoffigures % Write out the List of Figures

\lhead{\emph{List of Tables}} % Set the left side page header to "List of Tables"
%\listoftables % Write out the List of Tables

%----------------------------------------------------------------------------------------
%	ABBREVIATIONS
%----------------------------------------------------------------------------------------

\clearpage % Start a new page
\setstretch{1.5} % Set the line spacing to 1.5, this makes the following tables easier to read
\lhead{\emph{Abbreviations}} % Set the left side page header to "Abbreviations"
\listofsymbols{ll} % Include a list of Abbreviations (a table of two columns)
{
\textbf{ALU} & \textbf{A}rithmetic \textbf{L}ogic \textbf{U}nit \\
\textbf{ASIC} & \textbf{A}pplication \textbf{S}pecific \textbf{I}integrated \textbf{C}ircuit \\
\textbf{CPU} & \textbf{C}entral \textbf{P}rocessing \textbf{U}nit \\
\textbf{DAG} & \textbf{D}ata \textbf{A}ddress \textbf{G}enerator \\
}

%----------------------------------------------------------------------------------------
%	PHYSICAL CONSTANTS/OTHER DEFINITIONS
%----------------------------------------------------------------------------------------

%\clearpage % Start a new page
%\lhead{\emph{Physical Constants}} % Set the left side page header to "Physical Constants"
%\listofconstants{lrcl} % Include a list of Physical Constants (a four column table)
%{
%Speed of Light & $c$ & $=$ & $2.997\ 924\ 58\times10^{8}\ \mbox{ms}^{-\mbox{s}}$ (exact)\\
%Constant Name & Symbol & = & Constant Value (with units) \\
%}

%----------------------------------------------------------------------------------------
%	SYMBOLS
%----------------------------------------------------------------------------------------

%\clearpage % Start a new page
%\lhead{\emph{Symbols}} % Set the left side page header to "Symbols"
%\listofnomenclature{lll} % Include a list of Symbols (a three column table)
%{
%$a$ & distance & m \\
%$P$ & power & W (Js$^{-1}$) \\
% Symbol & Name & Unit \\

%& & \\ % Gap to separate the Roman symbols from the Greek

%$\omega$ & angular frequency & rads$^{-1}$ \\
% Symbol & Name & Unit \\
%}

%----------------------------------------------------------------------------------------
%	DEDICATION
%----------------------------------------------------------------------------------------

\setstretch{1.3} % Return the line spacing back to 1.3
\pagestyle{empty} % Page style needs to be empty for this page
%\dedicatory{For/Dedicated to/To my\ldots} % Dedication text
\addtocontents{toc}{\vspace{2em}} % Add a gap in the Contents, for aesthetics

%----------------------------------------------------------------------------------------
%	CONTENT - CHAPTERS
%----------------------------------------------------------------------------------------

\mainmatter % Begin numeric (1,2,3...) page numbering

\pagestyle{fancy} % Return the page headers back to the "fancy" style
\setstretch{1.0}
% Include the chapters of the thesis as separate files from the Chapters folder
% Uncomment the lines as you write the chapters

% Chapter 1
%\begin{multicols}{2}

\section{Introduction to Wireless Sensor Networks} % Main chapter title
\label{Section1} % To reference to this chapter elsewhere, use \ref{Section1} 
%\lhead{Chapter 1. \emph{Sound nomenclature}} % This is for the header on each page - perhaps a shortened title
%\textsf{\textsl{Written by Bjorn Deraeve}}
%----------------------------------------------------------------------------------------
%\begin{center}
%{\textsl{What if: }}
%\begin{large}
%\textbf{Ubiquitous Computing \& Networking} + \textbf{Sensing \& Control} ?
%\end{large}
%\end{center}
\subsection{Introduction}
The ENIAC, the first electronic computer designed by the American scientists J. Presper Eckert and John W. Mauchly, Turing-complete,  pioneering in 1946 and designed to calculate artillery firing tables \citep{FLAMM}. A wireless sensor network node, over 50 years later, 10 million times less the size of the ENIAC, still Turing-complete, and still military roots. The origins of the research to Wireless Sensor Networks (WSNs) can be traced \begin{wrapfigure}[9]{r}[\dimexpr.5\width+.5\columnsep\relax]{4cm}
  \centering
  \parbox{4cm}{\textsl{What if: }
\begin{large}
\textbf{\\Ubiquitous Computing \& Networking} \begin{center}+\end{center} \textbf{Sensing \& Control} ?
\end{large}}
  %\rule{5cm}{2.5cm}
\end{wrapfigure}back to DARPA\footnote{Defense Advanced Research Projects Agency (DARPA), is an American agency responsible for developing military technologies. DARPA has fund the research in many technologies which have had major effect on the world, including ARPANET, the first wide-area packet switching network (ancestor of the Internet and Douglas Engelbart's precursors to the GUI \citep{DARWIKI}.}, which sponsored a workshop at Carnegie Mellon University in 1978, identifying the technology components for a Distributed Sensor Network (DSN) \citep{DAR}. But at the time the technology was not quite ready. Sensors could take up the size of a shoe box and up, limiting the number of potential applications. The earliest DSNs were also not very tightly associated with wireless communication. In 1998 a new wave of research in WSNs started. Again DARPA acted as a pioneer by launching the initiative research program 'SensIT', which added new capabilities to the current sensor network such as ad hoc networking, dynamic querying and tasking, reprogramming and multi-tasking. At the same time IEEE noticed the high capabilities and low expenses of WSNs and defined the IEEE 802.15.4 standard for low power consumption, low data rate wireless PANs for 868MHz, 915MHz and 2.4GHz radios. Finally, in 2002, the ZigBee Alliance was established and published the ZigBee standard, based on IEEE 802.15.4. The standard adds a suite of high level communication protocols for WSNs such as device coordination, network topologies and interoperability with other wireless products.\\
Currently, WSNs are seen as one of the most prospective technologies of the 21st century \defcitealias{BW}{21 ideas for the 21st century, 1999}\citepalias{BW}. China for example, has involved WSNs in their national strategic research programmes (Program 973)\citep{CHINA}. The project follows an application-driven methodology and aims to issues identified with the real-world critical problems facing Chinese society. Over the years the program has funded areas such as agriculture, health, resources, energy, population and materials and brought significant benefits to China's sustainable economic and social development.
%----------------------------------------------------------------------------------------
\subsection{Wireless is everywhere}
\subsubsection{Definition of Wireless Sensor Network}
\emph{A Wireless Sensor Network (WSN) consists of spatially distributed autonomous sensors connected via a (wireless) communications infrastructure to cooperatively monitor, record and store physical or
environmental conditions, such as temperature, sound, vibration, pressure, motion or pollutants} \citep{SEBA}
%---------------------------------
\subsubsection{Internet of Things}
The Internet of Things (IoT) is a term that was first used by Kevin Ashton\footnote{Kevin Ashton co-founded the Auto-ID Center at the  Massachusetts Institute of Technology (MIT). The Center is a research group in networked RFID and newly emerging sensing technologies. The main goal was the development of the Electronic Product Code (EPC), a global RFID-based item identification system intended to replace the UPC bar code \citep{AUTO}} in 1999 and answers the question introducing this chapter. It refers to uniquely identifiable objects (things) and their virtual representations in an Internet-like structure. It is a vision of a network of Internet-enabled objects, together with web services which interact with these objects. If all objects in the world would be equipped with minuscule \begin{wrapfigure}[9]{l}[\dimexpr.5\width+.5\columnsep\relax]{4cm}
\vfill
\end{wrapfigure}identifying devices this could transform our daily lives. By embedding computational capabilities in all kinds of objects, including living beings, it will be possible to provide qualitative and quantitative shift in several sectors: logistics, domotics, healthcare, entertainment, and so on.\\
WSNs can provide a virtual layer where the information about the physical world can be accessed by any computational system. As a result, WSNs are one of the most important elements for realizing the vision of the Internet of Things paradigm \citep{ALCA}. On May 2nd, 2012, Libelium\footnote{The manufacturer of the Hardware we used. See chapter \ref{Chapter3} for more information.} published a list of 50 cutting edge \emph{Internet of Things} applications . According to Libelium, by 2020, more then 50 billion devices will be connected to the Internet \citep{50}.\\
The IoT can also be considered as the third wireless wave, following cellular technology and WiFi. Today wireless technology also includes the world of sense and control technology to bridge the gap between the virtual electronic world and our human physical world \citep{ATLANTIC}.
%---------------------------------------------------------------------------------------- 
\subsection{Characteristics of a WSN}
\subsubsection{Node architecture}
A typical sensor node (mote) has the following basic components:
\begin{itemize}
\item Microcontroller (+ memory)
\item Transceiver
\item Sensors (+ ADCs)
\item Power supply
\end{itemize}
The main controller options are Microcontrollers, DSPs, FPGAs or ASICs. Microcontrollers are the best option for a WSN node. They are general purpose and optimized for embedded applications, so they use little power consumption. DSPs are optimized for signal processing tasks and not suitable for WSNs. FPGAs can be good for testing purposes and ASICs are good if peak performance is needed but have no flexibility. The Libelium Waspmotes used for the test-bed developed for Group T have an 8-bit Atmel controller (see \ref{memory}).
%---------------------------------
\subsubsection{Fundamental Challenges in WSNs}
The biggest challenge WSNs encounter is without a doubt power consumption. Power! Power! Power! The next section will briefly introduce the general concepts and in section \ref{pow} the power consumption of the WSN we developed will be analyzed thoroughly. Other challenges often fall back to this limited amount of available energy, security for example.\\
There are dozens of other basic challenges for WSNs. Unattended operation and environmental influence makes a mote prone to failure. Mobility can cause topology changes. WSNs can use over more than 100 nodes, this leads to scalability issues and synchronization issues. There must also be a form of synchronization with sleeping nodes. Network responsiveness and robustness, not to forget making sense out of sensors.
%---------------------------------
\subsubsection{Power considerations}
\begin{figure}[htbp]
\centering
\includegraphics[width=0.3\textwidth]{typicalCons}
%\rule{30em}{0.5pt}
%\caption{Typical power consumption of a node}
%\label{fig:typicalCons}
\end{figure} 
%\vspace{1cm}
\begin{figure}[htbp]
\centering
\includegraphics[width=0.48\textwidth]{powerCons}
%\rule{30em}{0.5pt}
\caption{Typical power consumption of a node}
\label{fig:typicalCons2}
\end{figure}
\noindent Figure \ref{fig:typicalCons2} shows a typical power usage division in a WSN node \citep{NIWSN}. Clearly the main power cost is due to Transmission / Networking. As a result, to obtain an acceptable battery life nodes must sleep most of the time. Effective use of network transmission (section \ref{powerSaver} ), effective dynamic power management (section \ref{dynPow}) and optimize duty cycles are key to conserve power.\\
%---------------------------
\subsubsection{Benefits of Wireless Measurements}
Wireless Sensor Networks provide benefits in roughly three categories: installation and maintenance costs can be reduced, measurements can be optimized thus efficiency increases, and finally infrastructure limitations can be overcome.
%-------------------------------------------------------------------
\subsection{Wireless Standards and Technology}
\subsubsection{Standards enable growth: ZigBee}
Not only do standards allow devices from different vendors to interoperate, they also provide OEMs and integrators the flexibility of second sourcing. The ZigBee Alliance is an independent standardization organization which is driven by a large group of OEM companies and has definitely had a large impact on the rapid development of WSNs. Figure \ref{fig:stand} indicates the most critical properties of the ZigBee standard. Some rules-of-thumb are:
\begin{itemize}
\item The higher the frequency, the higher the data rate
\item The lower the frequency, the further the reach
\item All radio waves show strong absorption in water and metal 
\end{itemize}
\begin{figure}[htbp]
\centering
\includegraphics[width=0.48\textwidth]{zigbeeRange}
\caption{Comparing the ZigBee standard with Cellular, Bluetooth and WiFi}
\label{fig:stand}
\end{figure}
Table \ref{tab:range} shows the typical power consumption, throughput, range and application examples of each technology \citep{ZBWSN}.\\
\begin{table*}[!ht]
\begin{center}
\begin{tabular}{cc|c|c|c|l}
\cline{2-5}
 & \multicolumn{1}{ |c| }{\textbf{Battery Life}} & \textbf{Data Rate} & \multicolumn{1}{|c|}{\textbf{Range}} & \textbf{Application Examples}\\ \cline{1-5}
%\multicolumn{1}{ |c| }{Sleep duration} & High Performance & Power Saver & High Performance & Power Saver    \\ \cline{1-5}
\multicolumn{1}{ |c| }{\textbf{ZigBee}} & 1-4 years & 20 to 250Kbps & 100 m & Wireless Sensor Networks    \\ %\cline{1-5}
\hline
\multicolumn{1}{ |c| }{\textbf{Bluetooth}} & 1-2 weeks & 1 to 3 Mbps & 10 m & Wireless Headset   \\ %\cline{1-5}
\hline
\multicolumn{1}{ |c| }{\textbf{IEEE 802.11g}} & 1-2 days & 6 to 54Mbps & 30 m & Wireless Internet Connection   \\ %\cline{1-5}
\hline
\end{tabular}
\caption{Comparing the ZigBee standard with Cellular, Bluetooth and WiFi}
\label{tab:range}
\end{center}
\end{table*}
%---------------------------------
\subsubsection{ZigBee standard}
%----------------------------------------------------------------------------------------
At the moment ZigBee is the leading protocol to implement low-cost low-data-rate, short-range WSNs. It provides extra functionality regarding advanced routing capabilities and network stability. A common concept used to simplify and make digital communication more flexible, is the use of networking layers. Figure \ref{fig:stack} in appendix \ref{AppendixA} shows how this is organized in the ZigBee protocol stack.  The bottom two layers are defined by the IEEE 802.15.4 standard and define the specifications for PHY and MAC layers. ZigBee only defines the networking, application and security layer on top of IEEE 802.15.4.\\
%----------------------------------------------------------------------------------------
\section{ZigBee}
%----------------------------------------------------------------------------------------
\subsection{ZigBee standard}
%----------------------------------------------------------------------------------------
At the moment ZigBee is the leading protocol to implement low-cost low-data-rate, short-range WSNs. It is an open global standard built on IEEE 802.15.4 and provides extra functionality regarding advanced routing capabilities and network stability.
%-----------------------------------------
\subsection{ZigBee stack}
A common concept used to simplify and make digital communication more flexible, is the use of networking layers. Each layer is responsible for certain specific functions and pass data and commands only to the layers directly above or below via service access points. Figure \ref{fig:stack} shows how this is organized in the ZigBee protocol stack.  The bottom two layers are defined by the IEEE 802.15.4 standard and define the specifications for PHY and MAC layers. ZigBee only defines the networking, application and security layer on top of IEEE 802.15.4.\\
\begin{figure}[htbp]
\centering
\includegraphics[width=0.48\textwidth]{stack}
%\rule{30em}{0.5pt}
\caption{ZigBee Protocol Layers}
\label{fig:stack}
\end{figure} 
\paragraph{PHY Layer Responsibilities}
\begin{itemize}
\item Enabling and disabling the radio transceiver ... ...
\item Transmit and receive data ... ...
\item Select CH ... ...
\item Estimating signal energy ... ...
\item Providing RSSI, LQI ... ...
\end{itemize}
\paragraph{MAC Layer Responsibilities}
\begin{itemize}
\item Generating beacons ... ...
\item CSMA-CA, BPSK vs. O-QPSK, vs. ASK ... ...
\item Providing a reliable link ... ...
\item PAN association and disassociation services
\end{itemize}
\paragraph{Network Layer Responsibilities}
\begin{itemize}
\item Setting up a network ... ...
\item Allow joining and leaving a network ... ...
\item Configuring new devices ... ...
\item Discover and maintain routes ... ...
\end{itemize}
\paragraph{Application Layer Responsibilities}
\begin{itemize}
\item Application support ... ...
\item Address management and mapping ... ...
\item Define the role of the device ... ...
\item Security related tasks ... ...
\end{itemize}
%-----------------------------------------
\subsection{Networking concepts}
\subsubsection{Device Types}
\subsubsection{Device Roles}
\paragraph{Coordinator Operation}
\paragraph{Router Operation}
\paragraph{End Device Operation}
sleep...
\subsection{Parent - Child relationship}
%-------------------------------------------------------------------
\subsection{API Frame Specifications}
To communicate with the ZigBee radio there are 2 modes, AT (transparent mode) and API (application programming interface). AT mode means that what you send to the zigbee radio using RS232, the ZigBee radio will send to its default destination. Unless you send “+++”, wait for the ZigBee to reply with OK, and then send an AT command. AT commands are used to change the configuration of the ZigBee radio. For instance the AT command OP requests the operating PAN ID.\\
AT mode is fairly limited and only good for point to point communication since you can’t really specify the destination unless you change the default destination all the time. So that is why the sensor network operates in API mode. This means that everything sent to the zigbee radio, using serial communication, is now packetized.\\
API defines a number of different packets as can be found in chapter 9 of \defcitealias{XBEE}{XBee/XBee-Pro ZB RF Modules User Manual, 2012}\citepalias{XBEE}. An API packet is shown in figure \ref{fig:api}. It starts with 0x7E as start delimiter and is followed by the length of the data excluding checksum. Then a API-specific structure follow which depends on the type of packet.\\
\begin{figure}[htbp]
\centering
\includegraphics[width=0.48\textwidth]{api}
\caption{UART Data Frame Structure}
\label{fig:api}
\end{figure} 
\begin{table}[!ht]
\begin{center}
\begin{tabular}[!ht]{|c|c|}
\hline
\textbf{API Frame Name} & \textbf{API ID}\\
\hline
AT Command & 0x08\\
\hline
ZigBee Transmit Request & 0x10\\
\hline
AT Command Response & 0x88\\
\hline
ZigBee Receive Packet & 0x90\\
\hline
\end{tabular}
\caption{API Frame Names and Values}
\label{tab:apis}
\end{center}
\end{table}\\
\label{api1}\noindent
A reduced list of possible packets can be found in table \ref{tab:apis} (for the full list please see appendix \ref{AppendixF}). As mentioned an AT command is used to alter configurations of the ZigBee radio. This can of course also be done in API mode. For details about all the packets, please consult the datasheet. The only packets types used in this project are: 'ZigBee transmit request, 0x10' and 'Zigbee receive packet, 0x90'.\\
A ZigBee transmit request is shown in figure \ref{fig:api4}. This packet is used to send data from this ZigBee radio to a remote one. All that has to be known is the remote ZigBee address. These types of packets are constructed by the gateway to send out data to the libelium nodes but also by libelium nodes to send data to other libelium nodes or the gateway. Libelium has its own specific format for the RF Data as will be explained in section \ref{libPAQ}. To reach the gateway the address of the coordinator can be used, since the coordinator and gateway in our case are the same. This is convenient since the coordinator can always be addressed with 0x0000000000000000. The reason we chose the gateway and coordinator to be the same is that the coordinator receives a lot of traffic due to its role as coordinator and the same goes for the gateway. So these 2 devices should be in the center of the network for efficiency reasons. Assigning one device for these 2 roles and trying to position this device as central as possible will ask for the least amount of routing overhead.\\
When data is received by a ZigBee radio, this radio will send out a Zigbee receive packet via its serial communication. An example of this packet can be found in figure \ref{fig:api5}. Again the received data has an additional format as specified by Libelium.
\vfill  
%% Chapter 3
\chapter{A WSN with Waspmotes: Theoretical aspects} % Main chapter title
\label{Chapter3} % For referencing the chapter elsewhere, use \ref{Chapter1} 
\lhead{Chapter 3. \emph{A WSN with Waspmotes: Theoretical aspects}} % This is for the header on each page - perhaps a shortened title


\section{Introduction}
Waspmote is more than just another piece of hardware. In fact it is an open source platform for wireless sensors, specially focusing on low consumption and autonomy. Waspmotes promise to offer a variable lifetime between 1 and 5 years, depending on the duty cycle and the used radio.\\ 
But it didn't just start with Waspmote and it will definitely not end with it. In 2007 developers from Libelium collaborated with the Arduino Team creating the first open hardware shield for Arduino, the "Arduino XBee Shield". The shield allowed an Arduino board to communicate wirelessly via ZigBee. Libelium used the shield to develop their first sensor device, the SquidBee, intended for creating sensor networks. Although the SquidBee is self-powered and implements wireless communications, it is more sensor device than wireless sensor device. The 3.3V - 5V regulator could not be turned off, as a result there is a constant consumption of 50mA discharging the battery within hours. The SquidBee was created for teaching and educational purposes only. Since the platform was not radio certified the motes could not be deployed in real scenarios like cities, factories or even houses, so it did not fit Libelium's corporate customer requirements at all. However, the tone of an open hardware and source wireless sensor device was definitely set.
\begin{figure}[ht]
  \hfill
  \begin{minipage}[t]{.45\textwidth}
    \begin{center}  
      \epsfig{file=arduinoShield1, scale=0.20}
      \caption{Arduino XBee Shield}
      \label{fig:arduinoShield}
    \end{center}
  \end{minipage}
  \hfill
  \begin{minipage}[t]{.45\textwidth}
    \begin{center}  
      \epsfig{file=squidbee, scale=0.25}
      \caption{Libelium SquidBee}
      \label{fig:squidbee}
    \end{center}
  \end{minipage}
  \hfill
\end{figure}\\
In 2009 the Waspmote was born, meeting all the above requirements: low consumption and meeting three radio certification requirements (CE for Europe, FCC for the US and IC for Canada). In addition the Waspmote was built with a complete modular philosophy. The idea behind this design is to integrate only the needed modules in each device, optimizing costs. This is why all modules are connected to the Waspmote via sockets.\\Since its introduction, more than 2000 developers have been using Waspmote (v1.1) and the platform has received many suggestions and possible improvements. Libelium carefully listened to all these proposals and decided to bring out a new version with the name of Waspmote PRO (v1.2) in February 2013. This new version comes with upgraded hardware and an improved API, which is unfortunately not compatible with the older API. The most important improvements of the hardware is that the code can be uploaded much quicker and the XBee radio must no longer be removed, which is a huge improvement. There are also no more jumpers and there is no need of a coin battery. Regarding the API, Libelium claims it is much more robust and easier to use than the previous one and they also improved their programming guides. 
\begin{figure}[ht]
  \hfill
  \begin{minipage}[t]{.45\textwidth}
    \begin{center}  
      \epsfig{file=waspmote3, scale=0.45}
      \caption{Waspmote V1.1 with Bluetooth expansion module }
      \label{fig:arduinoShield}
    \end{center}
  \end{minipage}
  \hfill
  \begin{minipage}[t]{.45\textwidth}
    \begin{center}  
      \epsfig{file=waspmotePro, scale=0.40}
      \caption{Waspmote PRO}
      \label{fig:squidbee}
    \end{center}
  \end{minipage}
  \hfill
\end{figure}\\   
%----------------------------------------------------------------------------------------
\section{Hardware}
\subsection{Modular Architecture}
As mentioned in this chapter's introduction, Waspmote is based on a modular architecture, doing so optimizing costs and able to change according to the specific user's requirements. The available modules are split up into five categories:
\begin{itemize}
\item ZigBee
\item GSM - 3G/GPRS
\item GPS
\item Sensor Boards
\item Storage
\end{itemize}  
Figure \ref{fig:waspMote1} indicates the Waspmote main components.
\begin{figure}[ht]
\centering
\includegraphics[height=6.5cm]{waspmote2}
\rule{30em}{0.5pt}
\caption{Main Waspmote components}
\label{fig:waspMote1}
\end{figure}
%-------------------------------------------
\subsection{Microcontroller and memory}
Just like on any other PCB, the CPU is the heart of the module. Waspmote integrates an 8-bit ATmega 1281 microcontroller with 128KB programmable flash, 8KB SRAM runtime memory and 4KB EEPROM memory. Since SRAM is is built with cleverly combined MOSFETs it must not be periodically refreshed, but it is still volatile memory. The main advantage compared to DRAM is that, when moderately clocked like in the Waspmote, it consumes very little power.\\
Because of the modular design of the Waspmote the block diagram is very simple. This is shown in figure \ref{fig:block}.
\begin{figure}[ht]
\centering
\includegraphics[height=11.5cm]{block}
\rule{30em}{0.5pt}
\caption{Waspmote block diagram}
\label{fig:block}
\end{figure}
%-------------------------------------------
\subsection{Timers}
The Waspmote's system clock is an 8MHz quartz oscillator. This means that every 125ns a machine language instruction is executed by the microcontroller. Keep in mind that one C++ instruction consists of several instructions in machine language. To generate interrupts the Waspmote has an internal watchdog and a Real Time Clock (RTC).
\subsubsection{Watchdog}
The Watchdog is integrated on the Atmega 1281 and counts the clock cycles generated by a 128KHz oscillator. When the WDT counter reaches a set value it generates an interruption signal. The WDT is used to awake the microcontroller from \textit{Sleep} mode, because of its high precision. Thus, \textit{Sleep} mode allows small intervals, going from 16 milliseconds to a maximum of 8 seconds.
\subsubsection{Real Time Clock}
To store an absolute time base the RTC can be used. Alarms programmed in the RTC specify days, hours, minutes and seconds. For the RTC the waspmote uses a Maxim DS3231SN 32.768Hz oscillator. Because this clock has an internal compensation mechanism for variations caused by temperature changes this is one of the most accurate clocks on the market. This clock is used to wake the Waspmote from the higher energy saving modes \textit{Deep Sleep} and \textit{Hibernate}, with intervals from 8 seconds to even days. It is important to notice that in \textit{Hibernate}, the RTC is no longer powerd through the main battery but through the auxiliary (button) battery. So when problems occur when using hibernate probably it is recommended to measure the button battery's voltage and possibly must be replaced.
%----------------------------------------------------------------------------------------
\section{Programming}
To develop on the Waspmote, Libelium offers an API and IDE. For more information on the API, please see chapter \ref{Chapter4}. Waspmote uses the same IDE (compiler and core libraries) as Arduino, so as long as things like pin layout and I/O schemes are adjusted, code should be compatible in both platforms. So Waspmote and Arduino are pretty much the same, however don't forget that Waspmote has Radio Certifications and Arduino doesn't.\\  
When the Waspmote is started, the microcontroller will execute the bootloader and start loading the compiled program from FLASH into the SRAM working memory.\\
The code is divided into two basic parts: \textbf{setup} and \textbf{loop}, each with sequential behaviour. When the Waspmote is switched on or reset, the code starts at the setup function and then enters the loop function. Because the second part forms an infinite loop a common technique to save energy is to block the program until some interruption is detected.\\
Since in \textit{Hibernate} mode the Waspmote is completely disconnected from the main battery also the program is interrupted. This means the SRAM has lost all variable values and at wake up the code restarts at the \textbf{setup} function. To store values during hibernate cycles it is necessary to write them to EEPROM or the SD card.
%----------------------------------------------------------------------------------------
\section{Power considerations}
\label{pow}
\subsection{Waspmote power modes}
The libelium Waspmote has 4 operational modes: ON, Sleep, Deep Sleep and Hibernate. They differ from which type of interruptions they can be woken up and duration interval. For our application we want sleep intervals of 30 seconds and more, so only Deep Sleep and Hibernate mode are of interest. Table \ref{tab:cons1} summarizes the Waspmotes operational modes.
\begin{table}[!ht]
\begin{center}
\begin{tabular}[!ht]{|c|c|c|c|c|}
\hline
\textbf{Mode} & \textbf{Consumption} & \textbf{CPU} & \textbf{Cycle} & \textbf{Accepted Interruptions}\\
\hline
ON & 9mA & ON & - & Synch and Asynch\\
\hline
Sleep & 62$\mu$A  & ON & 31ms - 8s & Synch (WDT) and Asynch\\
\hline
Deep Sleep & 62$\mu$A & ON & 8s - min/hours/days & Synch (RTC) and Asynch\\
\hline
Hibernate & 0.7$\mu$A & OFF & 8s - min/hours/days & Synch (RTC)\\
\hline
\end{tabular}
\caption{Operational modes of Libelium Waspmote V1.1}
\label{tab:cons1}
\end{center}
\end{table}
\subsubsection{Deep Sleep}
In deep sleep mode the main program is paused and the CPU passes to a latent state. Triggers are as well synchronous interruptions (RTC) as asynchronous interruptions. Examples of asynchronous interruptions are low battery level or a sensor that reaches a certain trigger value.\bigskip
\begin{figure}[ht]
\centering
\includegraphics[height=3cm]{deepSleep}
\rule{30em}{0.5pt}
\caption{Waspmote going from ON to Deep Sleep}
\label{fig:deepSleep}
\end{figure}\bigskip
\\In figure \ref{fig:deepSleep} the process from going to operational mode ON to Deep Sleep is shown. The main advantage of this mode is that the program is only paused, so the program stack and thus all variable values keep their values. When the Waspmote is turned back on it simply executes the next instruction.
%-------------------------------------------
\subsubsection{Hibernate}
Hibernate mode consumes roughly 100 times less energy than Deep Sleep. This is made possible by disconnecting all the Waspmote's modules, including the microcontroller. The RTC gets his power through the auxiliary battery. So if hibernate mode stops working it is probably necessary to replace the Waspmote's button battery. Figure \ref{fig:hibernate} demonstrates the process from ON to hibernate.\\
\begin{figure}[ht]
\centering
\includegraphics[height=4.5cm]{hibernate}
\rule{30em}{0.5pt}
\caption{Waspmote going from ON to Hibernate}
\label{fig:hibernate}
\end{figure}
This means the CPU is also switched of and does not remember any values from variables. When waking up the Waspmote reinitializes, the microprocessor is reset and the program restarts from the beginning. Both \textbf{setup} and \textbf{loop} routines are executed as if the main switch would be activated. By placing the \verb+ifHibernate()+ function in setup the program can determine if it came from a hardware reset or from a hibernate reset. To be able to wake up from hibernate mode the hibernate jumper must be removed correctly. See section .... for remarks on this issue. \\Because not all Libelium's API functions regarding hibernate in combination with the different alarm modes work, it is advised to use the functions provided in \textbf{WaspXBeeZBNode.h}. 
%----------------------------------------------------------------------------------------
\subsection{Sampling sensors}
To measure the sensors, originally we took 10 samples with 100 milliseconds recommended delay between the measurements and calculated the average. Since we want to make the energy consumption as low as possible we now do the iterations without delay. Appendix \ref{apendixSensorSensitivity} contains 60 test samples per sensor and indicates that there is no significant difference on the average by removing this delay. Except for CO$_{2}$ measurements, removing the delay saves about 1000 milliseconds per measured sensor.\\ 
Because the first sample often shows a slight deviation, the program takes 11 samples but bases the average on the last 10 values.
%-------------------------------------------
\subsection{Battery life estimation}
In order to be able to give recommended sensor measuring intervals this section will analyse the estimated battery life of the Waspmotes. Table \ref{tab:cons2} enumerates the most common components typical consumption.
\begin{table}[!ht]
\begin{center}
\begin{tabular}[!ht]{|c|c|}
\hline
\textbf{Action} & \textbf{Average Current}\\
\hline
XBee, sending, & 105mA\\
\hline
CO$_{2}$ & 50mA\\
\hline
XBee, ON & 45mA\\
\hline
Waspmote, ON & 9mA\\
\hline
Pressure & 7mA\\
\hline
Humidity & 380$\mu$A\\
\hline
Waspmote, sleep & 62$\mu$A\\
\hline
Temperature & 6$\mu$A\\
\hline
Waspmote, hibernate & 0,007$\mu$A\\
\hline
\end{tabular}
\caption{Operational modes of Libelium Waspmote V1.1}
\label{tab:cons2}
\end{center}
\end{table}
The batteries included with our Waspmotes are rechargeable Lithium-ion batteries with a capacity of 6600mAH. The Waspmote that will be deployed on the roof of Group T, campus Vesalius will also have a 12V solar panel with a charging current up to 280 mA to extend its battery life. The other batteries can by charged manually or by USB (5V, 100mA). Lithium-ion have a self-discharge rate of typcially 1 to 2 percent per month and since the used batteries are new we expect a high battery efficiency.\\
%-------------------------------------------
\subsubsection{XBee and Waspmote start-up times}
For the XBee node to join an existing network there are two power related possibilities. Either the Waspmote has been turned on already sufficiently long and the XBee had more than enough time to join the network, or either the XBee wasn't joined yet and the program needs to wait on this. From the experiments done at our apartment we came to following conclusions:
\begin{enumerate}
\item It takes about 2.5 seconds to join a network after powered on.
\item If the XBee is joined, the program still needs to confirm this. This takes 452 milliseconds.
\item The sending time is constant, about 158 ms, if the XBee had more than 2.5 seconds to join. However in case the XBee must send immediately after it is joined, the sending time is not constant and takes on average 611 milliseconds.
\item The sending time increases if there are more obstructions between the antennas. 
\end{enumerate}
With these characteristics we came to results discussed in the next sections. For the validation of these conclusions please see appendix \ref{AppendixA}. Table \ref{tab:sendTime} sums up the results of the distance-relation test.
\begin{table}[!ht]
\begin{center}
\begin{tabular}[!ht]{|c|c|}
\hline
\textbf{Distance} & \textbf{Average sending time (ms)}\\
\hline
Air & 158\\
\hline
1 Floor & 268\\
\hline
2 Floors & 357\\
\hline
3 Floors & 484\\
\hline
4 Floors & 558\\
\hline
5 Floors & unreachable\\
\hline
\end{tabular}
\caption{Distance consequence on send times}
\label{tab:sendTime}
\end{center}
\end{table}\\
To save power the Waspmote can store the values for a user determined time. Taking samples and save them to EEPROM in case of hibernate mode takes only 6 - 7\% of the time to measure and send. Table \ref{tab:sendTime3} confirms this.
\begin{table}[!ht]
\begin{center}
\begin{tabular}[!ht]{|c|c|}
\hline
\textbf{Nr of samples per sensor} & \textbf{Average ON time (ms)}\\
\hline
10 & 210\\
\hline
3 & 194\\
\hline
\end{tabular}
\caption{Time needed to sample and store 4 sensors}
\label{tab:sendTime3}
\end{center}
\end{table}
%-------------------------------------------
\subsubsection{Battery life with standard program optimizations}\label{batLife1}
The application scenario for this battery test is as follow: the Waspmote will be turned on as short as possible and 4 sensors, namely temperature, humidity, pressure and battery level will be sampled. The node will take 10 samples for each sensor and calculate the average. Those values are put into one ZigBee packet and sent to the gateway. By adapting the sleep time between the event we came to the rather disappointing results shown in figure \ref{fig:batCalcHP}.\\ The graph in figure \ref{fig:batCalcHP1} breaks down the total energy consumption to five categories. It shows the monthly energy consumption as a function of the time between the events. For small intervals the active energy usage is huge. Only starting at 20 minutes sleep time the self-discharge becomes dominant and from 3 hours on the sleep mode current also becomes dominant.\\
Since the Waspmotes use this much energy when applied this way we will call this the High Performance mode from now on. The next section calculates an alternative approach, referred to as Power Saver mode. This nomenclature is continued in the program:
\begin{alltt}
    typedef enum \{HIGHPERFORMANCE, POWERSAVER\} PowerPlan; 
\end{alltt}
\begin{figure}[htbp]
\centering
\includegraphics[height=8.5cm]{batCalcHP}
\rule{30em}{0.5pt}
\caption{Battery life in High performance mode}
\label{fig:batCalcHP}
\end{figure}
\begin{figure}[htbp]
\centering
\includegraphics[height=8.5cm]{battery1}
\rule{30em}{0.5pt}
\caption{Energy usage in High performance mode}
\label{fig:batCalcHP1}
\end{figure}
Table \ref{tab:cons2} summarizes the battery duration in years of both performance and power saver mode.
\begin{table}[!ht]
\begin{center}
\begin{tabular}{cc|c||c|c|l}
\cline{2-5}
 & \multicolumn{2}{ |c|| }{Deep Sleep} & \multicolumn{2}{c}{Hibernate}\vline\\ \cline{1-5}
\multicolumn{1}{ |c| }{Sleep duration} & High Performance & Power Saver & High Performance & Power Saver    \\ \cline{1-5}
\multicolumn{1}{ |c| }{10s} & 0,05 & 0,92 & 0,05 & 1,00    \\ %\cline{1-5}
\hline
\multicolumn{1}{ |c| }{1min} & 0,21 & 2,15 & 0,21 & 2,63   \\ %\cline{1-5}
\hline
\multicolumn{1}{ |c| }{3min} & 0,79 & 2,75 & 0,85 & 3,59   \\ %\cline{1-5}
\hline
\multicolumn{1}{ |c| }{10min} & 1,25 & 2,85 & 1,39 & 3,76   \\ %\cline{1-5}
\hline
\multicolumn{1}{ |c| }{20min} & 1,75 & 2,90 & 2,06 & 3,85  \\ %\cline{1-5}
\hline
\multicolumn{1}{ |c| }{1h} & 2,41 & 2,94 &3,02 & 3,91    \\ %\cline{1-5}
\hline
\multicolumn{1}{ |c| }{3h} & 2,94 & 2,96 &3,90 & 3,94    \\ %\cline{1-5}
\hline
%\multicolumn{1}{ |c  }{\multirow{2}{*}{Powers} } &
%\multicolumn{1}{ |c| }{gcd} & 2 & 2 & 0 & 0 & min \\ \cline{2-6}
%\multicolumn{1}{ |c  }{}                        &
%\multicolumn{1}{ |c| }{lcm} & 3 & 3 & 1 & 1 & max \\ \cline{1-6}
\end{tabular}
\caption{Battery life in years for High Performance and Power Saver}
\label{tab:cons2}
\end{center}
\end{table}\\

%-------------------------------------------
\subsubsection{Battery life with extra optimizations}
As shown earlier in this section, sending values requires that the Waspmote is on for at least 3 seconds. In addition the XBee uses about five times the energy of the Waspmote. For end devices it is obviously recommended to turn on the XBee as little as possible, within a user defined limit.\\ Figure \ref{fig:batCalcPS} shows the same results as the application scenario discussed in section \ref{batLife1} and adds the results for a mode further referred to as Power Saver.\\
The implementation of this mode will depend on the nodes sleep settings. For \textit{Deep Sleep} the values can simply be stored on the heap, but for \textit{Hibernate} the values must be written to EEPROM.\\
Because of the size limit of a ZigBee packet we can store maximum 30 values and send them in one packet. However, if the sensor measuring interval is small the user can opt to store more values and send two or more packets after each other. The values for Power Saver in table \ref{tab:cons2} are of an example scenario that takes 60 measurements and then sends them in two packets to the gateway. It are also those results which are put in function of time in figure \ref{fig:batCalcPS}.\\

\begin{figure}[htbp]
\centering
\includegraphics[height=8.5cm]{battery2}
\rule{30em}{0.5pt}
\caption{Battery life High Performance vs. Power Saver}
\label{fig:batCalcPS}
\end{figure}\bigskip
\begin{figure}[htbp]
\centering
\includegraphics[height=8.5cm]{battery3}
\rule{30em}{0.5pt}
\caption{Energy usage in Power Saver mode}
\label{fig:batCalcPS1}
\end{figure}

As visible on figure \ref{fig:batCalcPS} \textit{Hibernate} has more influence in Power Saver mode, already extending battery life significantly at a 20 seconds interval comparing to a 10 minutes interval in High Performance mode. Also the energy breakdown graph in figure \ref{fig:batCalcPS1} shows that the interval times must be increased much less before the dominant factor is self-discharge and sleep mode energy consumption, compared to figure \ref{fig:batCalcHP1}.\\By reducing the sensor measurement accuracy battery life can be extended with modest 3 - 4\%, best case scenario. Please see appendix \ref{appendixA} for details.\\
In case the measuring intervals are small it is recommended to use \textit{Deep Sleep} instead of \textit{Hibernate}, since in hibernate the values are written to EEPROM. Equation \ref{eq:1} shows this can be very destructive for the Waspmote. Depending on how much freedom the user is given, the program can make the decision to switch to \textit{Deep Sleep} on itself, or the installation's administrator can control this.
%% Chapter 4
%----------------------------------------------------------------------------------------
\section{A WSN with Waspmotes: Implementation aspects}
\subsection{Introduction}
In this section the program running on the WSN nodes will be discussed. To start programming, Libelium offers its customers a customized IDE and a fairly extensive API. The IDE uses the same compiler (AVR) and core libraries as the Arduino IDE. The IDE is ideal to upload small examples and test programs. However, to facilitate programming and to obtain more C/C++ support, expanding the API is a better approach. This will be discussed more in section \ref{LibAPI}.\\
After experimenting with the ZigBee sleep modes it became clear that some unstable results caused by delayed pending messages for end devices could not be avoided. The program does support XBee sleep modes but this section will only discuss stable operation modes. From now on, we will no longer differentiate between ZigBee routers and ZigBee end devices, but the sleep options will be controlled via Waspmote sleep modes, as recommended by Libelium \defcitealias{LIB_SLEEP}{Libelium-dev, 2013}\citepalias{LIB_SLEEP}. So a ZigBee router forced to sleep as well as a ZigBee end device will be considered as an 'End Device'. 

%---------------------------------------------------

\subsection{Program Structure}
When ZigBee sleep modes are ignored, basically three different programs suffice to program the entire network. There is one program to support the gateway (which is also the ZigBee coordinator) to analyse the data received from the other nodes, which will be discussed in section \ref{extracting}. All the other nodes run either a 'Router' program or an 'End Device' program and are designed to collect and send data to the gateway. The only difference between these last programs is that a 'Router' program does not implement the sleep modes. Before continuing to the next sections, please have a look at the full 'End Device' program flowchart shown in figure \ref{fig:flow}.

\subsubsection{Initialization}
Figure \ref{fig:fsm} shows the simplified underlying logic of a dynamic C implementation of ZigBee at the application level. In one cycle each node has to associate with the network, measure some sensors, send the measured samples and possibly enter a sleep mode before repeating this cycle, depending on whether it has a 'Router' program or 'End Device' program running.\\ 
\begin{figure}[ht]
\centering
\includegraphics[width=0.35\textwidth]{fsm}
\caption{ZigBee Application State Machine}
\label{fig:fsm}
\end{figure}
\paragraph{Device start-up: Full initialization}
When the program is executed for the first time, a full XBee setup will be executed with the parameters entered via the Libelium IDE. To ensure stability it takes about 8 seconds to write the settings to the XBee, to reset it afterwards (turn the power off and back on) and finally to perform the joining process. From this moment on the node will send its battery level to the coordinator and wait to receive its physical sensor layout settings. Until such an 'ADD\_NODE\_REQUEST' packet is received the sleeping time will gradually be increased in order to save battery power.
\paragraph{Next cycles: Reduced initialization}
By not resetting the PAN ID but by fetching it from the XBee's memory, the joining process only takes about two seconds\footnote{By enabling ZigBee sleep the node does not lose the association with its parent and the re-joining process only takes about 10ms. However, the time needed to receive messages pending in router devices varies between 5 and 15 seconds and is never compensated by the faster joining process. There is also no guarantee that pending messages will be received or in which order they will be received, so network stability can no longer be guaranteed.} (Please see appendix \ref{fig:envA} for more measurement results). Unfortunately a disadvantage of this shortened setup is that the XBee is no longer able to detect if the coordinator or his parent is actually available. The result of the association check is not always correct and the program will only notice this for the first time when it is trying to send a message. If this function results in a send error the program will do a full setup routine and resend the message. If the node then fails again to send the message we can conclude that the coordinator is really off-line or that there are no joinable nodes within range. In that case the measured sensors will be saved and tried to send during the next cycles. In order not to lose results, a system administrator will be notified if a node fails to report for several consecutive cycles.

\subsubsection{Measuring sensors}
sensor float to bytes conversion... ...
\paragraph{Sending samples}
escaping zeros... ...
app ID = 10: 'IO\_DATA'... ...
see appendix ?? for all our packets... ...
\paragraph{Wait for received messages}
Since this mode does not use ZigBee sleep, end devices can only receive messages when the are turned on. To coordinate this, end devices will check messages for a fixed amount of time just after sensor data has been sent.\\
This means the Waspmote will only wake up when sensors have to be measured. Whereas in the next section ZigBee sleep is enabled and the Waspmote may also wake for polling, in order to avoid that messages buffered in routers get lost (since Waspmote v1.1 cannot be interrupted by available XBee RF Data). 


\subsubsection{Sending and Receiving}
\label{frames1}
In the API of Waspmote v1.1 Libelium uses an 'Application Header' which is shown in figure \ref{fig:appH} to send and receive data. This header takes care of packet fragmentation if packets exceed the maximum payload limit and can also be used by the receiver to treat the packet or fragment. The header itself is sent inside the RF Data field of the API Frame Structure (see figure \ref{fig:frames}).\\
\begin{figure}[ht]
\centering
\includegraphics[width=0.48\textwidth]{appHeader}
\caption{Application Header}
\label{fig:appH}
\end{figure}
The Waspmote and gateway programs use the 'Application ID' field in this header to distinguish different data in sent packets. In a first approach this extra layer we developed also serves as acknowledgement in the communication protocol. Depending on the 'Application ID' the layer contains a sensor mask is (a 16 bit-flag) and data. Examples are 'ADD\_NODE\_REQ' and 'CH\_SENS\_FREQ\_REQ'. Appendix \ref{OwnProtocolLabel} shows a complete overview off all options.\\
To easily switch between the different requests and select the sensors contained in the received mask the program uses static function pointers. An example is given below:
\usestyle{vs} %other useful styles are, bw,  borland, vs
% Include the source code 
\includecode{fp.cpp}

\subsubsection{Without ZigBee sleep mode}
%In this mode there is no difference between routers and end devices on XBee level. The only difference in program execution is that the sleeping period is skipped for routers.
The user can set the device role during setup or even later on by sending a command to the Waspmote, which makes it easy to change a mote's device role. The device role is stored in the xbeeZB device type identifier (see \defcitealias{ZBGUIDE}{Waspmote ZigBee Networking Guide, 2012}\citepalias{ZBGUIDE} and '\textbf{WaspXBeeZBNode.h}').
To better understand the next sections please have a quick look at the program flow chart in figure \ref{fig:flow}.





%According to Libelium their IDE has been properly tested and proven to assure optimum operation. Unfortunately we cannot agree with this. 
%Libelium also offers a lot of program examples on their website. Sadly most of them don't work like Libelium claims they do.
\subsection{AVR compiler}
\subsubsection{Toolchain Overview}
To develop software for an AVR microcontroller several tools are working together. This group of tools produces the final executable and is commonly called a toolchain and is shown in figure \ref{fig:tool}. 
\begin{figure}[ht]
\centering
\includegraphics[width=0.26\textwidth]{avr}
\caption{Overview of the AVR toolchain}
\label{fig:tool}
\end{figure}
\paragraph{GCC:} AVR uses the open source GNU Compiler Collection (GCC) with AVR microcontroller as target system. This version of GCC is known as 'AVR GCC' \defcitealias{AVR}{AVR-Libc User Manual, 2008}\citepalias{AVR}. GCC differs from other compilers, it only focuses on translating high level language to target assembly. For AVR GCC there are 3 language options: C, C++ and Ada. 
\paragraph{GNU Binutils: }The next step is done by another open source project called GNU Binutils. This contains the GNU assembler and GNU linker.
\paragraph{AVR-libc:} GCC and Binutils provide the tools to make the machine code but one critical component they do not provide is the Standard C Library. The open source AVR toolchain therefore comes with its own open source C Library project which contains many of the same functions found in the regular Standard C Library. It also adds many additional library functions that are specific to  AVR microcontrollers.
\paragraph{GNU Make:} Finally all pieces must be tied together. This is done by Make, which interprets and executes the Makefile of the project.
%------------------------------
\subsubsection{Memory Sections}
\label{memory}
The available non-volatile memory sections are the \textbf{.text} section (FLASH), which contains the actual machine instructions and the \textbf{.eeprom} section. Many AVR devices have a minimal amount of RAM. This limited amount of runtime memory needs to be shared between the following memory sections:
\begin{enumerate}
\item Initialized variables and static data such as\\ \verb+char message[] = "An error message"+ are stored in \textbf{.data variables}
\item Uninitialized global or static variables: \textbf{.bss variables}
\item Dynamic memory: \textbf{heap}
\item Area used for calling subroutines and storing local variables: \textbf{stack}
\end{enumerate}
The standard RAM layout is shown in figure \ref{fig:RAM}. Since there is no hardware supported memory management, separate regians can overwrite each ohter. Heap and stack can collide if either of them require large memory space or even when the allocations aren't high at all but because heap allocations get fragmented over time and new request don't fit in freed areas.
\begin{figure}[ht]
\centering
\includegraphics[width=0.48\textwidth]{ram}
\caption{AVR / ATmega1281 standard RAM layout}
\label{fig:RAM}
\end{figure}\\
As discussed in section \ref{memory} the ATmega1281 is uses a modified Harvard architecture, meaning that data can also be stored in program memory space. This is useful when you have constant data and you're running out of room to store it. Remember that many AVRs have limit amount of RAM to store data, but may have available FLASH space left. For the compiler this is however a challenge, which is exacerbated by the fact that the C Language was designed for Von Neumann architectures. So the AVR compiler has to use other means to operate with these separate address spaces (cf. pointer usage). The AVR toolset used the GCC \verb+__attribute__+ keyword, which is used to attach different attributes to function declarations and variables. AVR GCC provides a special attribute called \verb+progmem+ for data declarations and tells the compiler to store the data in Program Memory. To increase the convenience to the end user AVR-libc provides a simple macro \verb+PROGMEM+ which can be found in 'avr/pgmspace.h'. To read the data another macro is provided, which generates the correct address to retrieve the data from Program Memory. Storing data in Program Space incurs extra overhead in terms of instructions and execution time, but usually this is minimal compared to the space savings.\\
\subsubsection{Memory problems}
Libelium's Programming Style Guide warns its users about the amount of memory \verb+USB.print("Test message!")+ requires. The program memory increases due to the instructions and arguments (the chars) needed to print the string, since the message needs to be put in RAM memory first also there precious memory is lost (see the assembly extract in appendix ??).\\
Libelium recommends to do the following:
\usestyle{vs} %other useful styles are, bw,  borland, vs
% Include the source code 
\includecode{mem1.cpp}
This however still uses both program and data memory and is only useful if one wants to print the same message in different parts of the program, so this is not really a solution. The only way to save RAM memory while printing messages is to hard code them into the heap by doing the following:
\usestyle{vs} %other useful styles are, bw,  borland, vs
% Include the source code 
\includecode{mem2.cpp}
A less cumbersome way would be to give the message and an address where to store the message as argument of a recursive matrix which does this operation for us. However, standard C Language macro's cannot simply split a string into characters. So the only ways to save RAM is to hard code the string as data or to store the string in Program Space and use the \verb+strcpy_P+ to copy the string to stack when it is needed. 
\subsection{Libelium IDE and API}
\subsubsection{Waspmote-IDE}
The Libelium IDE offers some advantages compared to using other IDE's. For example by using Eclipse it is possible to update programs that are to big ( > 120KB ), over-writing the bootloader. Then they must be sent back to Libelium to restore them. The Waspmote IDE does not allow this accident, so Libelium does not support using other IDE's in an official way so that there is no valid warranty if you've erased the bootloader. Their IDE is far from perfect however, some issues we've experienced are:\\
\begin{itemize}
\item Opening a second or more instance of the IDE sometimes re-opens the previously active files, making it confusing to detect which one you were working in so you end up with two unsaved versions of the same code.
\item Once you start compiling (which takes a lot of time) there's no way to stop it, the stop button does not work.
\item Uploading immediately after compiling will first re-compile it anyway.
\item There is no complete C/C++ support. For example using simple enums is not possible. A workaround is to place the code in additional .h or .cpp files.
\item Auto-completion for the Libelium API functionality would be a great addition.
\end{itemize}
\bigskip
Also the Waspmote's (V1.1) hardware slows down the programming process:
\begin{itemize}
\item Uploading the code takes a lot of time: 1.5 - 2 minutes.
\item The uploading process fails if:
\begin{itemize}
\item The XBee is present
\item The hibernate jumper is not present when the mote is in hibernate
\item The little power switch has been turned off
\end{itemize}
\end{itemize}
Often you will want to turn off the power switch temporarily to analyse the content of the serial monitor. Especially in pair programming there is often one requirement you forget and the Waspmote does not check for this on beforehand. It will first compile and do as if it is uploading your code, disappointing you at the end of the process.\\
When debugging bigger program these actions come even more annoying. Suppose you are testing a program which measures sensors, sends the values and hibernates. Then you must:
\begin{enumerate}
\item Remove the sensor board
\item Place the hibernate jumper
\item Remove the XBee
\item Upload
\item Place the XBee
\item Remove the hibernate jumper
\item Re-mount the sensor board
\end{enumerate}
And this is not the end of the list. Removing the hibernate jumper causes the Waspmote to crash one out of two times. Resetting the mote has no effect in this case, just keep inserting and removing the little jumper until it agrees with what you want.
%------------------------------------
\subsubsection{Waspmote-API}
\label{LibAPI}
To facilitate the programming Libelium offers a quite big API and after some exploring you are quickly started with it. The structure of the API is very simple, there is little to no inheritance. In this section, the most important classes are indicated with a red box and are recommended to explore before starting future programming.
\paragraph{Original structure}
Each module or concept has its own class, for example all RTC functions are in 'WaspRTC.h' and 'WaspRTC.cpp'. To be able to use those functions in the IDE an object from the class 'WaspRTC' must be created. This object is created by default by the library and it is public to all libraries. All types to be run on the API can be found in the 'WaspClasses.h' file and each file also includes this file so it is aware of all available types. Please see appendix \ref{appendixC} for a complete overview.\\
For our application WaspXBeeZB is one of the most important classes. It inherits from WaspXBee Core and this way it is also related to the WaspXBee class. Figure \ref{fig:API2} displays the relationship with the AVR-libc libraries and the Waspmote hardware. For example \verb+typedef unsigned char uint8_t+ can be found in 'stdint.h'.\\ 
\begin{figure*}[ht]
\centering
\includegraphics[width=0.98\textwidth]{API2}
\caption{Reduced dependency of Waspmote core libraries}
\label{fig:API2}
\end{figure*}
\noindent During the development of our program the Waspmote showed several strange effects that could only be explained by bad stack management or heap and stack conflicts. Because of this lack of free memory (SRAM, 8KB) we discovered that the V1.1 API wastes a lot of memory by always including all libraries despite not using them. As a fix Libelium recommends to remove all classes you do not use, and there fields that are used in other classes, 'just' going through the compiler errors one by one. After this our program had enough free memory and showed normal behaviour.
%------------------------------------
\paragraph{Added functionality}
To facilitate programming extra functionality has been added to the Libelium API. They are inside files containing the original name with the 'Utils' addition, for example 'WaspRTCUtils.h' and can be found in 'BjornClasses.h'.
%----------------------------------------------------------------  




%----------------------------------------------------------------------------------------
\subsection{Sleep options}
After default configuration the Waspmote will send only its battery level to the default gateway, check for received commands and go into hibernate mode for 1 minute. In a first approach, which focusses on longer battery life, ZigBee sleep options are not taken into account. With ZigBee sleep enabled, routers and coordinators can buffer incoming RF data for their end device children. However, they can only do this up to 30 seconds \defcitealias{XBEE}{XBee/XBee-Pro ZB RF Modules User Manual, 2012}\citepalias{XBEE}. When an end device sleeps longer than 30 seconds, they should send a transmission when they wake to inform other devices that they are awake and can receive data. This is exactly what the first approach does, except it completely disconnects the XBee and sleep intervals are much longer.\\
To make the web service more reactive, this latency has been removed in the second approach. Here XBee sleep is enabled and end devices will respond to commands within 30 seconds.
In a last approach we have a look at how the energy usage of the first approach can be optimized even more.  
\subsubsection{Without ZigBee sleep}
This section discusses algorithms which can be used to measure sensors at variable times. It supposes the Waspmote uses either \textit{Hibernate} or \textit{Sleep} mode, completely disconnecting the XBee. Since it is not possible to combine RTC for both hibernate and sleep mode, a tweak has been implemented to use the Watchdog instead. This way the program can automatically chose which sleep mode to invoke, depending on the next duration to sleep. In hibernation mode the node is completely disconnected from the main battery and the program stops. This makes that all variables lose their values and must be stored in EEPROM memory. Each of the next techniques present with benefits and drawbacks and since we are working with embedded systems with limited possibilities, one should also consider to limit the users options to facilitate the calculations.
\paragraph{Calculate only the next time to sleep}
Each algorithm will have to store the individual sleep times per sensor. To support this algorithm also a copy of the original time will be saved and each time the node wakes up it will look for the smallest next time to sleep. This number will be subtracted from the other sleep times in the array. When a value becomes zero it will be restored with its original value and the cycle continues.  For an example which demonstrates this process please see appendix \ref{AppendixC}.\\
This process is fast and simple. However, the main advantage is that the node has to write to EEPROM each time it wakes up. According to the Atmel datasheet, the EEPROM of the ATmega1281 has an endurance of at least 100,000 write / erase cycles. The following equation indicates the problem for an interval of 10 seconds:
\begin{equation}
\frac{100000 \mathrm{writes} \cdot 10 \mathrm{s}}{60 \mathrm{s} \cdot 60 \mathrm{min} \cdot 24 \mathrm{h}}= 11,57 \: \mathrm{days} 
\label{eq:1}
\end{equation}
But the processors has 4Kbytes EEPROM on board so we don't have to write to the same place every time. Since EEPROM is written on a 'per cell' basis this can extend the lifetime. Our sensor mask can contain up to 16 values of 2 bytes. This leads to the next result:
\begin{equation}
\frac{100000 \mathrm{writes} \cdot 10 \mathrm{s} \cdot 4\mathrm{KB}}{60 \mathrm{s} \cdot 60 \mathrm{min} \cdot 24 \mathrm{h} \cdot 365\mathrm{days} \cdot 32\mathrm{B}} = 3,96\: \mathrm{years}
\end{equation}
We still must store where the data is stored but this won't cause big problems since we only have to rewrite this cell 125 times:
\begin{equation}
\frac{4\mathrm{KB}}{32\mathrm{B}} = 125 \: \mathrm{writes}
\end{equation}
\paragraph{Calculate all next times to sleep}
Another possibility is calculate as much as possible or maybe even all sleep necessary times. This algorithm first calculates the least common divider of the given measuring intervals. Afterwards memory is allocated to store the multiples of the values. When the LCM divided by the smallest measuring interval is smaller than the size of the allocated space, all values can be stored in EEPROM. If this is not the case also the last multiplier of the measuring interval must be stored, so when the last value is reached, the next series can be calculated.\\
Each time the Waspmote wakes up, it will compare its current RTC value with the times stored in EEPROM. The biggest stored time that is an integer multiple of the RTC value is the current position in the array. With this time the sensors to measure and the next time to sleep can be determined.
\paragraph{Limit user control}
Depending on the next time to sleep the program could decide for itself to go into hibernate, deepsleep or sleep with XBee sleep mode. Also, when the program detects inefficient measuring intervals, for example, 1 minute and 2 minutes 10 seconds, this can be notified to the installer or even be refused during setup.
%----------------------------
\subsubsection{With ZigBee sleep}
\paragraph{Managing End Devices}
ZigBee end devices are intended to be battery powered and are capable of sleeping for extended periods of time. Because of this, routers and coordinators use packet buffers and transmission timeouts to ensure reliable data delivery to end devices.\\
When an end device joins the network, a parent-child relationship is formed with a router or the coordinator. From then on, if the end device is awake, it will send poll request messages (by default every 100 ms) to its parent to determine if the parent has any data buffered for it, independent of the sleep mode. Routers buffer this data only up to 28 - 30 seconds, so if we want to ensure reliable communication, this is the maximum sleep time. The child poll timeout can however be set up to a couple of months, so an end device can sleep longer than 30 seconds and still be considered to be in the network. This includes the node is associated within a few milliseconds, compared to the 2.5 seconds mentioned in section \ref{startup}. End devices can choose between two sleep modes, discussed in the next sections. 
\paragraph{Pin sleep}
In this mode an external microcontroller controls when the XBee should sleep and when it should wake by controlling pin 13. The module will not respond to serial or RF data when it is sleeping.\\\\
\textbf{+} lowest power consumption\\
\textbf{+} external controller can take samples without powering up the radio\\
\textbf{-} ZigBee protocol has less control\\
\textbf{-} external controller's timer is not accurate enough to synchronize the network\\
\textbf{-} Need fully awake parent\\
\paragraph{Cycle sleep}
Allows the XBee to determine when to wake up. The module can sleep for a specified time and wake for a short time to poll its parent for buffered data. If the parent has date the device will remain awake for a time, otherwise it will re-enter sleep mode immediately. 
\begin{itemize}
\item + suitable for DigiMesh, where the sleep clocks is accurate enough to get all nodes awake at the same time
\item + with DigiMesh, fully awake routers are not required, so they can be battery powered
\item - more power consumption due to accurate clock
\item - external controller must also wake when the XBee wakes to treat potentially received messages, even if there is no need to sample data
\end{itemize}

\paragraph{XBee sleep parameters}
\textbf{Router/Coordinator Configuration:} 
\begin{itemize}
\item RF Packet Buffering Timeout: 
\begin{itemize}
\item Sleep Period (SP) parameter. Max 30 seconds.
\item For cyclic sleep devices: SP should be set the same on routers and coordinators as it is on cyclic sleep end devices
\item For pin sleep devices: SP should be set equal to the time the end device can sleep, up to 30 seconds. If an end device sleeps longer than 30 seconds, parent and possibly non-parent devices must know when the device is awake. Therefore and devices that sleep longer than 30 seconds should transmit some kind of data (API frame) to alert the other devices that they can send data to the end device.
\end{itemize}
\item Child Poll Timeout: Sleep Number (SN) parameter: The number of Sleep Periods (SP) used to calculated end device poll timeout.

SP = 30
SN = 3E8
\end{itemize}
... ...
%--------------------------------------------------------------------

 
%% Chapter 5
%----------------------------------------------------------------------------------------
\section{Extracting the data}
\label{extracting}
\subsection{Programming language choice}
To start programming a choice had to be made. Which programming language and what operating system? Since we only knew Java, C++ and C these languages had our preference. Another requirement was that running the program on an embedded device. Linux has more options is the embedded domain then windows. For instance the raspberry pi is a cheap embedded device with an Ethernet connection and 2 usb ports and a 700mhz ARM processor that can run different linux distributions.\\
The easiest language is definitely java. It is easy because a lot of libraries are already included in the language. For a gateway several libraries are necessary. A HTTP library to send requests to ipsum and receive requests directly from the web application. Second, an XML or JSON library to format data in these HTTP requests. Since the cloud storage system, Ipsum, uses XML the rest of the gateway also uses XML instead of JSON.\\
On the other side there is the zigbee network which communicates using RS232. In java this serial communication is also build in. In C or C++ it depends on the OS, for linux it is possible to use read and write to and from file descriptors.\\
So on first sight java seems to be the preferred choice. But since it has to run embedded and processing power might be limited java is less ideal. On a raspberry pi computing power is no problem. Maybe in the future more services should run on the same device so there is an advantages in using C or C++.\\
So this leaves C and C++. They are both equally fast in execution but C++ is easier to develop since  it support more libraries (libraries for C van also be used in C++ but not the other way around) and it has a lot more typechecking so that code that compiles will probably also execute without errors. In C there is a lot of static casting to void pointers and this often results in harder to find bugs. C++ also supports objects and classes that make it easier to create a larger program. There are several other advantages in C++ over C. An entire chapter is dedicated to the advantages of C++ over C in the book “thinking in C++” by Bruce Eckel. Also we wanted to have some more experience in C++ and writing this program in C++ sure improved our programming.
%-------------------------------------------------------------------
\subsection{Implementation aspects}
For the gateway it took some time to come up with the correct structure. Since a gateway generally has to wait a lot for incoming messages on different connections it is logical to construct a lot of threads which can wait while other threads keep on running. An option would be to use the pipeline pattern where 1 thread “generates” packets. This would be the zigbee receiver or the webservice. Then a few threads would act as filters on these packets, doing the necessary processing.\\
One simple pipe would not work since there are more threads generating packets. For instance the ipsum thread can also generate packets to indicate that some packets could not be processed correctly. If ipsum is down this thread needs to store messages in the sql database.\\
There is not really one flow of information going from one	 place to another. Information is coming in at different places and is also leaving the program at different places.\\
A more flexible pattern is the thread pool pattern. This pattern is often used for webservers. Each thread then handles an incoming connection. Although mongoose uses this pattern in this way, it is not entirely the same for the gateway program. The gateway has a few threads, one for each type of connection. 
%---------------------------------
\subsubsection{Ipsum}
There is one thread for ipsum where the connection to the ipsum database is maintained. Uploading data, changing frequency or in use setting of sensors are all tasks this thread has. The ipsum thread has an incoming queue and an outgoing queue. The incoming queue can contain several packets to indicate what data should be uploaded, what sensor frequencies should change or what nodes should be activated. If for some reason something goes wrong, ipsum will push packets on the outgoing queue to the main thread.
%---------------------------------
\subsubsection{ZigBee}
On the zigbee side there are 2 threads, one for sending and one for receiving packets. Received packets are pushed onto a queue that is read out in the main thread. Received packets can be sensor data, errors and responses from sent packets. When the main thread needs something to be done in the zigbee network it will push a packet onto the queue going to the zigbee sender thread. For instance: changes sample frequencies, request IO data, activate a node.
%---------------------------------
\subsubsection{Web service}
As mentioned before, mongoose is used as webserver. Mongoose sets up some threads to handle incoming connections. All these threads can push packets onto a queue going to the main thread. In the main threads the XML data is analyzed and the right actions are performed. Possible webservice requests are: add node, add sensor, request IO and change frequency.
%---------------------------------
\subsubsection{Main thread}
All intelligence can be found in the main thread. There it is decided what to do with incoming packets. Most often this means creating packets and putting them in the appropriate queues. \\
The main thread also has access to local storage in the form of an sqlite database. Node information has to be stored locally in order to link zigbee addresses to the correct Ipsum ID’s. When a sensor data packet is received the main needs to look up to what ipsum sensors it has to upload this data. The relation between Ipsum ID and zigbee address is made when the webservice receives add node and add sensor requests.\\
The local database also stores Ipsum packets that could not be sent. This could happen when the connection to Ipsum is down or the Ipsum service has crashed. Since losing data is not acceptable, it is also stored in in the sql database.\\
Also errors could be logged in the database. For now errors are printed to \verb+cerr+. 
%------------------------------------------------------------------
\subsection{Hardware}
The gateway consists of a zigbee radio connected to a computer via RS232. This computer could evolve into an embedded linux device such as the raspberry pi (http://www.raspberrypi.org/). It is important that it runs on linux since some libraries are necessary and the serial communication is based on system calls to the kernel. For instance xerces for XML parsing or boost for the multithreading and some other small features.\\
The library for the webserver is mongoose and the one for en sql database is sqlite. Both libraries are written in C and consist out of 1 header and 1 source file and are both compiled into the final program. So unlike xerces and boost the OS will not need to install these libraries since they are not dynamically linked.\\
RS232 is a simple serial protocol that is used for low data rates. In linux an RS232 connection is easily set up by opening a file descriptor with the necessary options to configure baud rate, parity, stop bits, etc. After that you can use read and write functions to receive and send data from the zigbee radio.

 
%\section{Discussion}
\section{Future Work}
add security
\section{Conclusion}
\section{Acknowledgements}
 
%% Chapter 7
\chapter{Communication with LabVIEW User Interface using UART Port Controller} 
\label{Chapter7}
\lhead{Chapter 7. \emph{Communication with LabVIEW User Interface using UART Port Controller}}
\textsf{\textsl{Written by Thuy Pham}}
%----------------------------------------------------------------------------------------
\section{Introduction}
Communicating with peripheral devices plays a vital role in system design in general and in DSP systems specifically. For this purpose, SHARC Processor uses Digital Peripheral Interface (DPI). The interface provides both connections to two serial peripheral interface ports (SPI) and two universal asynchronous receiver-transmitters (UARTs). In this report, Universal asynchronous receiver transmitters (UARTs) will be examined in relationship with LabVIEW User Interface.\\
It is necessary to notice that the processor provide a full duplex universal asynchronous receiver/transmitter (UART) port, which is fully compatible with PC-standard UARTs. The UART port provides a simplified UART interface to other peripherals or hosts, supporting full duplex, DMA supported asynchronous transfers of serial data. The UART also has multiprocessor communication capability using 9-bit address detection. This allows it to be used in multi-drop networks through the RS-485 data interface standard. The UART port also includes support for five data bits to eight data bits, one stop bit or two stop bits, and none, even, or odd parity. More specifically, the UART port supports two modes of operation as mentioned below:
\begin{itemize}
\item \textbf{PIO (programmed I/O):} The processor sends or receives data by writing or reading I/O mapped UART registers. The data is double-buffered on both transmit and receive.
\item \textbf{DMA (direct memory access):} The DMA controller transfers both transmit and receive data. This reduces the number and frequency of interrupts required to transfer data to and from memory. The UART has two dedicated DMA channels, one for transmit and one for receive. This is the mode that will be used for transferring data between LabVIEW program and DSP board.
\end{itemize}
%----------------------------------------------------------------------------------------
\section{UART data frame}
\begin{figure}[htbp]
\centering
\includegraphics[height=3cm]{uart}
\rule{30em}{0.5pt}
\caption{A typical UART data frame}
\label{fig:uart}
\end{figure}
When two UARTs communicate, both transmitter and receiver need to know the signaling speed. The receiver does not know when a packet will be sent (no receiver clock); thus, the protocol is termed "asynchronous." This is very important because if we set up parameters (baud rate, start bit, stop bits, and parity bit) in both transmitter and receiver is incompatible, we will get incorrect data. In addition, the receiver circuitry is correspondingly more complex than that of the transmitter. The transmitter simply has to output a frame of data at a defined bit rate. Contrastingly, the receiver has to recognize the start of the frame to synchronize itself, and therefore determine the best data sampling point for the bit stream.\\ 
The processor supports a set of parameters' values for UART communication as below:
\begin{center}
\begin{tabular}[t]{|c|c|c|}
\hline
\textbf{SST} & \textbf{Parameters} & \textbf{Value}\\
\hline
1 & Baud rate & 2400, 4800, 9600, 19200, 38400, 57600, 115200, 921600, 6250000\\
\hline
2 & Data bits & 5 to 8 \\
\hline
3 & Stop bits & 1 or 2 \\
\hline
4 & Parity & None, even, odd \\
\hline
\end{tabular}
\end{center}
%----------------------------------------------------------------------------------------
\section{UART external interface}
The DSP processor communicates with peripheral devices in DSP board is described as the block diagram in figure \ref{fig:uart1}:
\begin{figure}[htbp]
\centering
\includegraphics[height=8.5cm]{systemArchitectureDPI}
\rule{30em}{0.5pt}
\caption{System Architecture Block Diagram}
\label{fig:uart1}
\end{figure}\\
\begin{figure}[htbp]
\centering
\includegraphics[height=6.5cm]{uart2}
\rule{30em}{0.5pt}
\caption{Schematic for connection of UART port}
\label{fig:uart2}
\end{figure}
In this board, the EIA-232E interface is used to communicate with other external serial communication devices. Instead of using traditional circuit for compatible purpose of power levels, in the DSP board, manufacture uses ADM3202 chip. The significant features of the chip are low power consumption and can operate at data rates up to 460 kbps make them ideal for battery powered portable instruments and high speed requirement.
%----------------------------------------------------------------------------------------
\section{UART configuration in DSP Processor for communication}
The DSP Processor provides a set of PC style, industry standard control and status registers for the UART.\\
\textbf{Register Overview for UART Module:}
\begin{itemize}
\item Line Control Register (UARTxLCR). Controls format of the data character frames. It selects word length, number of stop bits and parity.
\item Divisor Latch High/Low Register (UARTxDLL, UARTxDLH). Characterize the UART bit rate. The divisor is split into the divisor latch low byte (UARTxDLL) and the divisor latch high byte (UARTxDLH).
\item Mode Control Register (UARTxMODE). Controls packing and address modes.
\item Transmit Buffer Control Register (UARTxTXCTL). Controls core or DMA operation.
\item Receive Buffer Control Register (UARTxRXCTL). Controls core or DMA operation.
\item Interrupt Enable Control Register. Enables interrupt requests from system handling.
\item Line Status Register (UARTxLSR). Returns status of controls format of the data character frames as overrun or framing errors and break interrupts.
\item Transmit Status Register (UARTxTXSTAT). Returns status of core or DMA operations.
\item Receive Status Register (UARTxRXSTAT). Returns status of core or DMA operations.
\item Interrupt Identification Status Register (UARTxIIR). The register is used to get the status of all interrupts into one channel.
\end{itemize}
To configure parameters for transmitting data between PC and the DSP board, the number of simple steps can be followed:
\begin{enumerate}
\item \textbf{Route the UART to the DPI of DSP Processor}\\
The routing is implemented by software, in particular in the VisualDSP++, it is in the SRU macro. It is necessary to notice that, data is transmitted and received by the least significant bit (LSB) first (bit 0) followed by the most significant bits (MSBs).
\begin{figure}[htbp]
\centering
\includegraphics[height=6.5cm]{uart3}
\rule{30em}{0.5pt}
\caption{UART Functional Block Diagram}
\label{fig:uart3}
\end{figure}
\item \textbf{Set up baud rate, data bits, stop bits, parity bit}\\
The bit rate is characterized by the peripheral clock (PCLK) and the 16-bit divisor. The divisor is split into the UART divisor latch low byte register (UARTxDLL) and the UART divisor latch high byte register (UARTxDLH).  UART Baud rate = PCLK/(16 � divisor), where PCLK is the system clock frequency.\\
All data words require a start bit and at least one stop bit. With the optional parity bit, this creates a 7 to 12-bit range for each word. The format of received and transmitted character frames is controlled by the line control register (UARTxLCR).
\item \textbf{Set up core mode of operation}\\
This requires software management of the data flow using either interrupts or polling.\\
Core transfers move data to and from the UART by the processor core. To transmit a character, load it into the UARTxTHR register. Received data can be read from the UARTxRBR register. The processor must write and read one character at time. To prevent any loss of data and misalignments of the serial data stream, the UART line status register (UARTxLSR) provides two status flags for handshaking - UARTTHRE and UARTDR.\\
Core transfers through the UART is started by setting up and writing to transmit and receive control registers, enabling the module using the UARTEN bits in the UARTxTXCTL and UARTxRXCTL registers.
\item \textbf{Set up the UART interrupt}\\
The UART receive and transmit interrupts are programmed through the peripheral interrupt control registers (PICRx) as separate interrupts. (By default, these interrupts are not configured in the IRPTL register - the PICRx register has to be programmed to configure them.) \\
The UART interrupt enable register (UARTxIER) is used to enable requests for core system handling of empty or full states of UART data registers. When polling is used as a means of action, the UARTRBFIE and/or UARTTBEIE bits in this register are normally set.
\end{enumerate}
%----------------------------------------------------------------------------------------
\section{Programming for communication between DSP Processor and LabVIEW User Interface}
\subsection{LabVIEW program for writing data} \label{sec:Pham}
LabVIEW program provides a set of powerful tools for programming User Interface. Based on these things, user can make a program that clearly and visually shows parameters, graphs and results. LabVIEW also has many instrument drivers, which are a set of modular software functions that use the instrument commands or protocol to perform common operations with the instrument, for a variety of programmable instruments that use the GPIB, VXI, PXI, or serial interfaces.\\
With serial communication, in particular in RS-232, which is a standard developed by the Electronic Industries Association (EIA), LabVIEW provides some VISA functions for the target. They are located in \textbf{All Functions $>>$ Instrument $>>$ I/O $>>$ Serial}
\begin{figure}[htbp]
\centering
\includegraphics[height=6cm]{uart4}
\rule{30em}{0.5pt}
\caption{Write/Read a string to and from a COM port}
\label{fig:uart4}
\end{figure}
The VISA Configure Serial Port VI initializes the port identified by VISA resource name to the specified settings: timeout sets the timeout value for the serial communication; baud rate, data bits, parity, and flow control specify those specific serial port parameters.
\begin{enumerate}
\item The VISA Configure Serial Port VI initializes the port identified by VISA resource name to the specified settings: timeout sets the timeout value for the serial communication; baud rate, data bits, parity, and flow control specify those specific serial port parameters.
\item The VISA Write function sends the string.
\item The VISA Read function reads back up to bytes into the buffer, and the Simple Error Handler VI checks the error condition.
\item The VISA Close function terminates the communication channel to the instrument and deal locates the resources for the DSP.
\end{enumerate}
Depending on the purpose, the string can be written to the device in different forms. As the subVI described below, the string is combined by three parts, which are identified parameter order (two characters), the value needed to pass, and the character that uses for identifying read or write (character "w"). It is important to notice that maximum character transmission rate is depended on the baud rate and the bits per frame. More specifically, this rate is just the baud rate divided by the bits per frame. So, when there is a consecutive transfer required, the number of characters per string needs to be cared. Otherwise, there will be data lost.
\begin{figure}[htbp]
\centering
\includegraphics[height=4.8cm]{uart5}
\rule{30em}{0.5pt}
\caption{Formatting a string before writing to DSP}
\label{fig:uart5}
\end{figure}
Based on the prime subVI as showed in figure \ref{fig:uart5}, there are a number of ways to program an application that writes and reads multiple parameters to and from a DSP Processor. It needs to be reminded that, the principle of transferring data between DSP and PC is based on the interrupt. In other words, every time data is written to the DSP from PC, the serial DSP interrupt occurs and then its ISR (interrupt service routine) will run and process the received data. As a result, it is impossible to write more than one parameter at the same time that the DSP can process independently. Of course, each parameter has a different purpose. Because of this, the parameters should be organized so that only one parameter can write its data to the DSP at the time. It is very important for copying data between parameters that will be studied later.\\
Figure \ref{fig:uart6} shows an example program that wants to write multiple parameters to the DSP.\\
\begin{figure}[htbp]
\centering
\includegraphics[height=6.6cm]{uart6}
\rule{30em}{0.5pt}
\caption{A part of writing parameters to Operator 1 program}
\label{fig:uart6}
\end{figure}\\
The idea of the program is that every time a parameter changes its value, which is detected due to search 1D array, the read/write part starts working. Otherwise, the program does nothing. This reduces a lot of works for the DSP Processor. Then, the latest value of the parameter and its order will be combined into string before writing to the processor. Notice that, LabVIEW also provides some data convert functions so that different data types of parameters can work properly.
%----------------------------------------------------------------------------------------
\subsection{UART Interrupt}
There are three things needed to be cared while using UART interrupts. First, you must map the UART interrupts to one of the interrupts in the interrupt vector table. In particular, the UART has to be mapped to exactly the peripheral interrupt sources. This can be achieved by changing the default source of the peripheral interrupt priority control register with the UART source, using the interrupt select values of the UART receive interrupt. The receive interrupt select values for UART0 (using in our application program) and UART1 are 0x13 and 0x14, respectively.
\begin{center}
\begin{tabular}[t]{|c|c|c|}
\hline
IIUART0RX & 0x3E00 & Internal Memory Address for UART0 Receiver\\
\hline
\end{tabular}
\end{center}
\begin{figure}[htbp]
\centering
\includegraphics[height=1.5cm]{uart7}
\rule{30em}{0.5pt}
\caption{PICR2 Register}
\label{fig:uart7}
\end{figure}
\begin{alltt}
   *pPICR2 &= ~(0x3E0);  // Sets internal memory address for UART0 receiver
   *pPICR2 |= (0x13<<5); // Sets default value (0x13) for the URAT0 receive interrupt
\end{alltt}
The second thing is that enabling the UART interrupts internally by setting the corresponding bits in the UART interrupt enable register (UARTxIER). This needs to be done after all the UART settings (such as word length, parity, and so on) have been programmed, because in transmit mode as soon as the transmit buffer empty bit is enabled in the UARTxIER register, it vectors to the interrupt. If this bit is enabled before any of the UART settings are programmed, the data transmitted in the transmit interrupt service routine does not comply with the UART settings that are programmed later, leading to a communication error.
\begin{alltt}
   *pUART0IER = UARTRBFIE;   // Enables UART0 receive interrupt
\end{alltt}
The third, which is also the most important, is that using C Interrupt Handler with own Interrupt Service Routine. C provides its own set of interrupt handlers via the interrupt() and signal() functions. VisualDSP++ has extended the interrupt() function with \textbf{interruptf()} and \textbf{interrupts()} versions. The interrupt handler consists of three parts - the initialization, the interrupt vector, and the interrupt service routine. 
\begin{alltt}
   interrupt(SIG_P13, UARTisr);
\end{alltt}
Initialization performs the tasks previously associated with the C interrupt() function - assigning the interrupt service routine to an interrupt vector (dynamic ISR vectors only), unmasking the interrupt, and enabling global interrupts. The interrupt vector routes execution to the appropriate interrupt service routine. This routine must be compatible with the procedure of writing data of LabVIEW program. Otherwise, the data that DSP Processor expected to receive is different with data transmitted.
\begin{alltt}
   // Read the data from the buffer
   // Receive Buffer Registers (UARTxRBR)
      value = *pUART0RBR;
 
   /* Echoes back the value on to the hyperterminal*/
   // Wait until it is sent
      while ((*pUART0LSR & UARTTHRE) == 0)\{ ; \}

   // Writting a string to DSP
   // The character in buffer is "w" (ASC-II of "w" = 0x77 equivalent to 119 decimal)
      if(value == 119)
      \{
         for(i = 0; i < 2; i++)  address[i] = inputstream[i]; 
      
         // The two first characters writting from LabView is the index
         // int atoi(const char *str); - Convert string to integer 
            index = atoi(address);

         // The rest of the writting string is value (with a character - w)
         // Double atof ( const char * str ); - Convert string to double
         // 2 - is the place of starting value string 
            writtenValue = atof(inputstream + 2);

         // Writting value to the global variable
            algoparam[index] = writtenValue;    

         // Adds data to the buffer
         // Transmit Holding Registers (UARTxTHR)
            *pUART0THR = value;

            teller = 0;
      \}
\end{alltt}
%----------------------------------------------------------------------------------------
\subsection{The FM synthesizer application}
The application is programmed to provide two services. First, the program can write the values of parameters to the DSP so that DSP board can produce different sounds. Secondly, the LabVIEW User Interface also can show a number of features of the signal, such as the form of the envelope, frequency domain and the spectrogram.\\
To provide the features above, the application is programmed following a hierarchy as figure \ref{fig:uart8}.
\begin{figure}[htbp]
\centering
\includegraphics[height=5cm]{uart8}
\rule{30em}{0.5pt}
\caption{VI Hierarchy of LabVIEW User Interface}
\label{fig:uart8}
\end{figure}
\begin{figure}[htbp]
\centering
\includegraphics[height=5cm]{uart9}
\rule{30em}{0.5pt}
\caption{Front Panel of the application}
\label{fig:uart9}
\end{figure}
There are four main parts programmed in the application, which are programs for different algorithms of combining operators, programs for the envelope, programs for equalizer, sound effect and display, and writing data to the DSP board. The first part includes four algorithms: two operators, three operators, four operators modulated straightforward and four operators that actually consist of a pair of two operators. This part can be expanded in future to have more possibilities for producing sounds.
\begin{figure}[htbp]
\centering
\includegraphics[height=5cm]{uart10}
\rule{30em}{0.5pt}
\caption{The algorithm for three operators}
\label{fig:uart10}
\end{figure}
The second part of the application is the envelope. It is used to modify the shape of the signal so that the output can have different frequency components. The third is programmed to produce more flexibility for the application. With this, user can choose specific range of frequency of the output signal (using the equalizer) and also can make distortion (using the effect part). The last one is writing data part, which is discussed in ''LabVIEW program for writing data", see section \ref{sec:Pham} above. More detail about the application, subVIs is available in appendix part in this report.
%----------------------------------------------------------------------------------------
\section{Conclusion}
This report discusses a number of things, which are needed to take into account when working with UART controller of the DSP processor and LabVIEW program. From my point of view, despite the fact that each part plays a different role to make the application work properly, the most important thing is that the data (a string of characters) need to be define completely the same in LabVIEW User Interface and UART interrupt service routine. Otherwise, the data that DSP Processor wants to receive for processing is not expected.\\ To improve quality of communication between LabVIEW User Interface and DSP board, context switch of UART interrupt needed to be considered. However, because of time and the limited ability, it could be studied in the future.
 
%% Chapter 8

\chapter{Critical reflections} 
\label{Chapter8}
\lhead{Chapter 8. \emph{Critical reflections}}
\textsf{\textsl{Written by Bjorn Deraeve}}
%----------------------------------------------------------------------------------------
\section{Implementation problems}
In chapter \ref{Chapter6} we already explained how the program on the SHARC processor is designed. Here we will have a closer look at some problems encountered during the development.
\subsection{Communication with the Labview interface}
As explained in chapter \ref{Chapter7}, the parameters read from the interface are interrupt driven. As a consequence the program on the SHARC processor does not receive values if the user doesn't change any particular input on the interface.\\
For normal use this is not a problem since the values are initialized correctly when the program is started. However during the debugging process this often led to confusion and searching on problems in the code that actually weren't there.
%----------------------------------------------------------------------------------------
\subsection{Combining different program parts and general programming issues}
Including smaller topics like the envelopes, effects and equaliser into the main program sometimes led to major problems. Next to the inevitable minor adjustments in the program parts also the interaction with the interface caused problems. However, most of these problems were small mistakes that could easily be fixed. Nevertheless not all team members succeeded into fixing those little errors.\\
Sometimes it was necessary to change some details of the main program in order for the subprograms to work fluently. However, in order to do so efficiently one must at least understand the template we received at the beginning of the project and think well before changing stuff. There had to be intervened several times because a team member unknowingly stopped the whole program instead of just disabling his own algorithm. For example, as explained in chapter \ref{Chapter6}, processors on embedded systems need a \verb+while(TRUE)+ loop. Another useful thing to realize is that the periodic functions generating the waveforms need a variable that keeps track of the time. Updating this time in a local variable does not have the same effect like updating the global variable and again stops the program!\\
Also the modulo \%-operator is common in DSP algorithms and all team members should its meaning. Looking this up with google would have been a great idea!\\ Finally before asking to include a subprogram into the main program the smaller part must first have been debugged for syntax errors (since notepad doesn't highlight such errors).\\Next, some other concrete programming issues are described:
\begin{description}
\item[Sampling frequency:] The template we started from mentionned a sampling frequency of 48 kHz. However that project used a stereo audio channel which meant the samples were sent to the AD1835A at a rate of 96 kHz. Because we filled the algorithm's frames with mono channel samples (instead of two samples per calculation) our sampling frequency in software matched the sampling frequency of the hardware. \\ \\
\item[Modulation:] Enabling modulation caused some minor timing problems, however this could simply be fixed by passing the function's argument by value instead of by reference. \\ \\
\item[Envelopes:] There were several problems with the envelope. In a first stage nothing worked and a trick must be implemented to create envelopes with a duration longer than one second. To do so the function received a counter in its arguments. \\
The updating of this counter was at first done in the main synthesize sequence, when the program's time was an integer multiple of the sampling frequency. However because of the fixed frame length of 1024 samples this happend only every 4 seconds. To avoid this the updating of counter was originally moved to the create signal function. This has no influence and even made things worse actually. The problem with placing the flag here was that if modulation is enabled the flag updated several times. However this whole time the counter argument worked and the envelope refused working even without modulation.\\The underlying cause was that the envelope function made use of the wrong envelopes in memory so the parameters had no influence. Unfortunatelly not all team members were able to find such mistakes\\
Finally to fix all little problems the envelope received its own space in memory for keeping track of time. So now this isn't done anymore in the synthesize function but in the envelope itself. Because now the time is obviously updateted instaneously all problems are fixed. 
%\item[Envelopes, the bigger problem: ] There were several problems with the envelope. In a first stage nothing worked
\end{description}
%----------------------------------------------------------------------------------------
\subsection{Connecting the DSP board and use of the IDEs}
There were numerous of problems with Analog Devices' IDE. Personally I could not connect with the board on my normal Windows. Running the program from within a virtual machine on that same computer seemed to magically solve the problem (after a few years of internetting). If we have to invent an explanation I'd say that Sony tampered with the OS' USB drivers to make them more energy efficient and thus not supporting the DSP board.\\
Getting VisualDSP++ and Labview to run fluently also took a lot of time. For VisualDSP++ an activation code had to been registered. This became a problem since by the end of the project we got warnings that the activation would expend in less than a few days. Another issue with this activation code was that in week 10 a team member asked for it while it had already been e-mailed to all team members in week 4.\\
VisualDSP++ also seemed to be a very hard program to work with since a team member needed step-by-step click instructions for how to activate the editor's line numbers. \\
For Labview there were no valid licenses at all so the program had to be reinstalled a few times. Also to use all aspects of the interface several extra Labview plugins needed to be installed and again the only way to do is by completely reinstalling Labview, this became more or less a routine.
%----------------------------------------------------------------------------------------
\subsection{Extreme programming}
Due to the tricky situation with the interface several problems asked a lot of time to get fixed or didn't get fixed at all. For those harder to find errors a second pair of eyes would have been welcome.\\
Though in an attempt to improve the overal impression of the project it was best to do have look at those extra problems and happily been able to fix some of them. \emph{However it cannot be that there is only one person responsible to make all things work properly, the lack of a good programming partner was big. With such a partner a lot of improvements could have been made, some of them are discussed in section \ref{sec:future}.}
%----------------------------------------------------------------------------------------
\subsection{Latex}
Latex' standard color package does not support the color pink. My disappointment was huge, after heavy consideration I will declare this as the biggest lack encountered during this EE5-project.  
%----------------------------------------------------------------------------------------
\section{Team problems}
Next to common problems teams sometimes encounter like poor communication, group thinking and difficulty making decisions we also suffered from other problems. They are described shortly below:
\begin{itemize}
\item \textbf{Major issues}
\begin{itemize}
\item Lack of basic programming knowledge
\item Update files to the shared folder
\item Lack of inspiration for simple program features
\end{itemize}
\item \textbf{Minor issues}
\begin{itemize}
\item Update timesheets in time
\item Labview vs. C distinction
\item Give feedback on the programs: this only happened for the interface
\item Chaotic organisation of files in the shared folder
\end{itemize}
\end{itemize}
%----------------------------------------------------------------------------------------
\section{Achieved activities}
\begin{table}[hbp]
\begin{center}
\begin{tabular}[htbp]{| >{\centering\arraybackslash}m{4cm} | >{\centering\arraybackslash}m{4cm} |}
\hline
\textbf{Student} & \textbf{Hours}\\
\hline
Bjorn Deraeve & 210\\
\hline
Frederik De Greef & 140 \\
\hline
Kenneth Piot & 140 \\
\hline
Thuy Pham & 180 \\
\hline
\end{tabular}
\caption{Project logbook}
\label{tab:logs}
\end{center}
\end{table}
\begin{table}[!htbp]
\begin{center}
\begin{tabular}[!htbp]{|c|c|c|}
\hline
\textbf{Student} & \textbf{Topic}\\
\hline
Bjorn Deraeve & Admin: Meeting reports\\ 
& Labview: Connect the basic interface\\
& C: General program structure \\
& C: Signals and modulation \\
& C: Import and adapt (fix) the other programs\\ 
& C: Debug and fix envelope program\\
& Matlab: plots for report\\
& Labview: Bessel functions\\
& \LaTeX: cls, bib and tex template, chapters 1,2,5,6,8\\
\hline
Frederik De Greef & C \& Matlab: effects \\
& C \& Matlab: equaliser\\
& Report: chapter 4\\
\hline
Kenneth Piot & C: Envelopes \\
& Labview: Piano interface \\
& Report: chapter 2\\
\hline
Thuy Pham & Labview: full interface \\
& C: Configuration of UART communication \\
& Labview: waves and specta \\
& Labview: envelopes and graphs \\
& Labview: equaliser \\
& Labview: piano interface \\
& Report: chapter 7\\
\hline
\end{tabular}
\caption{Activities}
\label{tab:acts}
\end{center}
\end{table}
%----------------------------------------------------------------------------------------
\section{Future work}\label{sec:future}
This section summarizes some features that could add additional value to DSP Synthesizer.
\begin{description}
\item[Envelopes:] It is no coincidence the piano interface is located in the chapter on the envelopes. Combining the piano keystroke with the envelopes should have been an easy to implement extension.
\item[Equaliser:] An extended equaliser would provoke the musicians' creativity more.
\item[SDRAM:] Being able to save created sounds and reuse them to modulate signals. Also prestored effects could be offered this way. Search how to initialize the chip, \dots
\item[SHARC-ADSP 21369 IRQs:] Using the board's buttons to increase or decrease the volume. This is an interrupt driven event. Combining this interrupt with the other, time pressured interrupts, would have presented some interesting programming problems.
\item[Hardware controller:] Instead of using the labview interface use a real controller with a more advanced communication protocol. The board supports SPI, I$^{2}$C,\dots
\end{description}
 

%----------------------------------------------------------------------------------------
%	BIBLIOGRAPHY
%----------------------------------------------------------------------------------------

\label{Bibliography}

\lhead{\emph{Bibliography}} % Change the page header to say "Bibliography"

\bibliographystyle{ieeetr} % Use the "unsrtnat" BibTeX style for formatting the Bibliography

\addtocontents{toc}{\vspace{2em}}
%\clearpage
%\addcontentsline{toc}{\vspace{1em}}
\addcontentsline{toc}{chapter}{Bibliography}{\vspace{1em}}
\nocite{*}
\bibliography{./Bibliography} % The references (bibliography) information are stored in the file named "Bibliography.bib"




%----------------------------------------------------------------------------------------
%	CONTENT - APPENDICES
%----------------------------------------------------------------------------------------

\addtocontents{toc}{\vspace{2em}} % Add a gap in the Contents, for aesthetics

\appendix % Cue to tell LaTeX that the following 'chapters' are Appendices

% Include the appendices of the thesis as separate files from the Appendices folder
% Uncomment the lines as you write the Appendices

% Appendix A

\chapter{Bessel functions and spectrum examples} % Main appendix title

\label{AppendixA} % For referencing this appendix elsewhere, use \ref{AppendixA}

\lhead{Appendix A. \emph{Bessel functions and spectrum examples}} % This is for the header on each page - perhaps a shortened title

\section{Modulation index I = 0}
\hspace{1cm} $J_{0}$(0) = 1 (carrier component) \hspace{1cm} $J_{n}$(0) = 0 for n $\geq$ 0  (sidebands) \\ 
\begin{figure}[htbp]
\centering
\includegraphics[height=8cm]{besselspec0}
\rule{30em}{0.5pt}
\caption[Spectrum of a 1000Hz signal]{Spectral diagram of a pure 1000Hz signal. Naturally there are no sidebands.}
\label{fig:besselspec0}
\end{figure}\\
\begin{figure}[htbp]
\centering
\includegraphics[height=8cm]{bessel0}
\rule{30em}{0.5pt}
\caption[Bessel function of order 0]{Bessel function of the carrier component (order 0) Remark the amplitude is equal to 1 for I = 0: $J_{0}$(0) = 1}
\label{fig:bessel0}
\end{figure}\\
\begin{figure}[htbp]
\centering
\includegraphics[height=8cm]{bessel0sideband1}
\rule{30em}{0.5pt}
\caption[Bessel function of order 1]{Bessel function of the first sidebands component (order 1) Remark the amplitude is equal to 0 for I = 0: $J_{1}$(0) = 0}
\label{fig:bessel0}
\end{figure}\\
\section{Modulation index I = 1}
\hspace{1cm} $J_{0}$(1) $<$ 1 (carrier component begins to decline) \\ 

\hspace{1cm} $J_{n}$(1) $\geq$ 0 (sidebands begin to increase)  
\begin{figure}[htbp]
\centering
\includegraphics[height=8cm]{besselspec1}
\rule{30em}{0.5pt}
\caption[Spectrum of 1000Hz modulated with 100Hz signal]{Spectral diagram of a 1000Hz signal modulated with a 100Hz signal and I = 1.}
\label{fig:besselspec1}
\end{figure}
\begin{figure}[htbp]
\centering
\includegraphics[height=8cm]{bessel0carrier}
\rule{30em}{0.5pt}
\caption[Bessel function of order 0, I = 1]{Bessel function for the carrier (order 0) and I = 1 shows that the amplitude of the carrier component decreases.}
\label{fig:bessel0carrier}
\end{figure}
\begin{figure}[htbp]
\centering
\includegraphics[height=7.5cm]{bessel11}
\rule{30em}{0.5pt}
\caption[Bessel function of order 1, I = 1]{Bessel function of the first sideband component (order 1) returns an amplitude of 0.4 if I = 1. Compare this value with figure \ref{fig:besselspec1}}
\label{fig:bessel11}
\end{figure}
\begin{figure}[htbp]
\centering
\includegraphics[height=7.5cm]{bessel21}
\rule{30em}{0.5pt}
\caption[Bessel function of order 2, I = 1]{Bessel function of the second sideband component (order 2) returns an amplitude of 0.1 if I = 1. Compare this value with figure \ref{fig:besselspec1}}
\label{fig:bessel21}
\end{figure}
\begin{figure}[htbp]
\centering
\includegraphics[height=3cm]{besselLabview}
\rule{30em}{0.5pt}
\caption[Bessel functions in Labview]{Simple labview program to display Bessel functions of the first kind}
\label{fig:besselLabview}
\end{figure}

%input{./Appendices/AppendixB}
%% Appendix C

\chapter{Block diagram of SHARC ADSP-21369 processor} % Main appendix title

\label{AppendixC} % For referencing this appendix elsewhere, use \ref{AppendixA}

\lhead{Appendix C. \emph{Block diagram of SHARC ADSP-21369 processor}} % This is for the header on each page - perhaps a shortened title

\begin{figure}[htbp]
\centering
\includegraphics[height=11cm]{sharc.png}
%\rule{30em}{0.5pt}
\caption{Block diagram of SHARC ADSP-21369 processor.}
%\label{fig:sharc}
\end{figure}

%% Appendix D

\chapter{Pin Buffers} % Main appendix title
\label{AppendixD} % For referencing this appendix elsewhere, use \ref{AppendixA}
\lhead{Appendix D. \emph{Pin Buffers}} % This is for the header on each page - perhaps a shortened title

\begin{figure}[htbp]
\centering
\includegraphics[height=5cm]{pinbufferOutput}
\rule{30em}{0.5pt}
\caption[Pin buffer Output]{Pin Buffer as output: pin buffer enable is logically high, the amplifier acts as a current source and drives the value at pin buffer input onto the DAI/DPI pin. Remark that pin buffer output is the same as pin buffer input and can be driven as an output.}
%\label{fig:sharc}
\end{figure}
\begin{figure}[htbp]
\centering
\includegraphics[height=5cm]{pinbufferInput}
\rule{30em}{0.5pt}
\caption[Pin buffer Input]{Pin Buffer as input: pin buffer enable is logically low, the amplifier disables (high output impedance) and an off-chip source can drive a value to the DAI/DPI pin via pin buffer output. In this case pin buffer input is not used.}
%\label{fig:sharc}
\end{figure}
It is recommended programming the pin buffer input logic low when the pin buffer enable is logic low. The pin buffer enable is by default connected to a SPORT pin enable signal which may change in value. By making the pin input buffer low the line is decoupled from possible irrelevant signals and code becomes easier to debug.

%% Appendix D

\chapter{Labview: key functions} % Main appendix title
\label{AppendixE} % For referencing this appendix elsewhere, use \ref{AppendixA}
\lhead{Appendix D. \emph{Labview: key functions}} % This is for the header on each page - perhaps a shortened title

\section{Envelope program}

\begin{figure}[htbp]
\centering
\includegraphics[height=12cm]{envprog}
\rule{30em}{0.5pt}
\caption{Envelope program}
%\label{fig:sharc}
\end{figure}
\newpage
\section{Clipping program}

\begin{figure}[!h]
\centering
\includegraphics[height=8cm]{clipping}
\rule{30em}{0.5pt}
\caption{Clipping program}
%\label{fig:sharc}
\end{figure}

\section{Equaliser program}

\begin{figure}[!htbp]
\centering
\includegraphics[height=9cm]{equaliser}
\rule{30em}{0.5pt}
\caption{Equaliser program}
%\label{fig:sharc}
\end{figure}


\addtocontents{toc}{\vspace{2em}} % Add a gap in the Contents, for aesthetics

\backmatter


\end{document}  
